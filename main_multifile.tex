\documentclass[promaster]{thesis-uestc}
\title{基于深度学习的极化SAR图像目标分类方法研究}{Research on Target Classification Method of Polarimetric SAR Images Based on Deep Learning}

\author{林小惟}{Xiaowei Lin}
\advisor{杨建宇\chinesespace 教授}{Prof. Jianyu Yang}
\school{信息与通信工程学院}{School of Information and Communication Engineering}
\major{ 通信工程(含宽带网络、移动通信等)}{Communications engineering}
\studentnumber{202122011004}
\setdate[oral]{2024年4月15日}
\setdate[submit]{2024年3月15日}
\setdate[confer]{2024年6月8日}
% require all the usepackages here
% \usepackage{algorithm2e}

\begin{document}

\makecover
% \thesisfigurelist % 图目录,仅在需要时添加,一般情况下请注释
% \bindpdfcover{cover.pdf} %用于提交最终版论文的独创性声明,即有电子签名
\originalitydeclaration %用于提交审阅版本时的独创性声明,即无电子签名

% This is a template of mutiple files.
% The folders chapters/ and misc/ have the related files

% abstract
\begin{chineseabstract}
    合成孔径雷达(Synthetic Aperture Radar, SAR)是一种对地成像雷达,具有全天时、全天候对地物目标进行持续探测的特性。极化SAR在SAR的基础上,通过发射、接收不同极化方式的电磁波信号,实现对地物目标的极化散射特性探测,具有更丰富的极化信息。极化SAR图像解译技术目前已经广泛应用于军事探测、灾害分析、城市规划等多个领域。

    随着极化SAR技术的不断发展,高分辨、多极化的SAR图像数据日益增多,为复杂场景分析提供了更丰富的地物极化信息。丰富的极化信息、增加的数据量为极化SAR解译带来了积极作用,但是也带来了新的挑战。丰富的极化信息要求特征表示方法能高效联合不同类型极化信息。此外,极化SAR数据集标注过程中引入的标签噪声要求分类方法能在存在错误标记下进行精确的分类。本章针对极化信息提取与标签噪声问题,提出了适用于极化SAR的信息提取方法与鲁棒的目标分类方法。

    本文的主要研究工作如下:

    (1)开展标签噪声下极化SAR图像分类研究之前,一个性能优越的极化信息提取方法是有必要的。针对多类型极化信息表征问题,提出了一种基于双通道注意力的极化信息提取方法。该方法根据极化SAR散射特征与目标分解特征的差异性,利用双通道注意力的网络结构,独立提取极化关键信息。在此基础上通过信息联合,对双通道的两种类型极化信息进行整合,挖掘两类特征间的互补信息。此外,利用跨空间特征学习方法,聚合不同尺度极化特征。联合以上方法,实现地物目标极化信息更好的表征,从而提升下游分类任务精度。实验证明,相比于使用单类型极化特征或直接叠加极化特征,本方法提升了分类任务准确度。

    (2)在上述研究基础上,针对极化SAR标签噪声问题,提出了基于混合模型估计与边界增强的鲁棒性分类方法。该方法根据噪声样本、干净样本在深度网络模型中损失的分布差异,基于贝塔混合模型对损失分布进行拟合,实现样本噪声概率估计。同时,基于Sobel算子对极化SAR的Pauli伪彩图进行边界提取并膨胀,对膨胀边界内部的样本进行损失增强。最后,对自学习损失函数改进,通过样本噪声概率动态调整预测与标签之间的权衡,实现鲁棒性的学习优化。

    本文提出的方法通过了真实极化SAR数据集验证,实验结果验证了这些方法在极化信息提取与标签噪声下目标分类的有效性,提升地物目标分类任务准确率。



    % 本文研究工作
    % 总结


    \chinesekeyword{极化合成孔径雷达,极化SAR图像,极化信息提取,地物目标分类,标签噪声}
\end{chineseabstract}



\begin{englishabstract}
    Polarimetric Synthetic Aperture Radar (PolSAR) is an active microwave remote sensing technique known for its all-weather, all-day imaging capabilities. Due to its multi-channel, multi-polarization working characteristics, PolSAR images contain rich scattering information of targets. PolSAR image interpretation techniques have been widely applied in various fields such as military detection, disaster analysis, and urban planning.

    In recent years, the increasing availability of high-resolution, multi-polarized SAR image data has provided richer polarization information of ground targets for complex scene analysis, which has positively impacted PolSAR interpretation efforts but also introduced new challenges. Directly stacking all types of polarization information leads to information redundancy, while single-type polarization information is inadequate for all classification targets. This necessitates effective utilization of multi-type polarization information in feature representation methods. Additionally, label noise introduced during PolSAR dataset annotation requires classification methods to accurately classify samples with partial mislabeling. This thesis addresses the aforementioned issues through relevant fundamental research and innovations, with the following main research contents:

    (1) Investigation of PolSAR-related theoretical foundations. Firstly, characterization methods of electromagnetic wave polarization properties are introduced. Subsequently, several methods for describing target scattering characteristics and various polarization matrices are discussed. Finally, the theoretical explanation of polarization target decomposition methods is presented, along with a summary of the physical meanings of different polarization parameters.

    (2) Addressing the problem of redundancy in multi-type polarization feature information, a PolSAR target classification method based on dual-channel attention is proposed. By constructing a dual-channel PolSAR feature input network structure and employing spatial and channel attention modules, as well as multi-scale learning methods, different types of polarization features are refined and mutual information is extracted. Moreover, aggregation of polarization features at different scales is performed, effectively avoiding the limitations of single features and reducing the redundancy of multi-feature information, thereby improving PolSAR target classification accuracy.

    (3) In response to label noise issues, a robust PolSAR target classification method under label noise is proposed. By establishing the relationship between the loss distribution differences of noisy and clean samples under a deep neural network model, the noise probability estimation of samples is obtained. Additionally, combined with boundary sample loss enhancement, robust parameter optimization under label noise is achieved. Furthermore, full utilization of boundary information is implemented, resulting in improved PolSAR target classification accuracy under label noise.

    The effectiveness and superiority of the proposed methods are validated using real PolSAR data. Results demonstrate that these methods effectively address the problems of redundancy in multi-polarization feature information and label noise in PolSAR image classification, achieving high-accuracy classification of ground targets.

    \englishkeyword{Polarimetric Synthetic Aperture Radar (PolSAR), PolSAR image, polarization feature, target classification, label noise}
\end{englishabstract}




% table of contents
\thesistableofcontents

% thesis contents
\chapter{绪\hspace{6pt}论}

\section{课题研究背景及意义}
合成孔径雷达(Synthetic Aperture Radar, SAR)作为一种对地成像雷达,通过搭载在飞机或卫星上实现对地物目标测量。与传统的光学传感器件相比,SAR探测成像不受时间、气候等因素的影响,具有全天时、全天候的成像特点,可以实现持续对地物目标的探测\citing{皮亦鸣2007合成孔径雷达成像原理}。全极化SAR作为SAR系统的子类,通过发射接收不同极化方式的电磁波信号,实现对地物目标散射信息的探测。相比于传统的SAR系统,极化SAR数据具有更丰富的散射特征,蕴含更多的目标信息。目前,由于极化SAR的多极化特性,极化SAR技术已经成为合成孔径雷达技术中不可或缺的分支。在极化SAR的众多应用中,极化SAR图像解译工作已经成为其中最重要的研究方向之一,目的是快速、准确地获取极化SAR图像中的地物目标信息,在军事目标探测、城市规划、灾害估计等多个军事民事领域已经得到了广泛应用\citing{pramudya2019estimation,gao2018ship}。

极化SAR图像分类技术在解读极化SAR数据方面扮演着重要的角色。其核心在于特征提取和目标分类方法等技术的实施。可靠的地物目标分类结果依赖于合理且有效的特征提取工程和目标分类方法,才能充分体现出极化SAR复杂探测数据的重要价值。由于极化SAR的多极化成像特性,极化SAR图像中具备了丰富的极化信息,而这些信息的发现与利用需要借助有效的极化特征表示方法。特征表示是极化SAR图像解译中最基本、最关键的步骤,对最终的分类结果具有重要的影响。利用合理的特征表示,能够使分类模型更加快速、准确的迭代到最优参数,取得优秀的分类结果;反之,不合理的特征表示即便是使用更加复杂的分类器模型,也很难达到出色的分类结果。极化SAR图像的目标分类算法也属于解译的关键步骤,其本质是为极化SAR图像中的每一个像素赋予合理的类别。目标分类算法旨在确定一个非线性的分类规则,将所有具有相似特征的样本划分为同一类别,而将不相似特征的样本归为不相同的类别。有效分类同类和异类地物目标是分类器设计的关键所在。

近年来,相比于极化SAR成像能力方面的持续进步和显著发展,极化SAR图像解译技术的提升较为缓慢,这也导致了极化SAR在实际工程中应用的限制。首先,由于成像系统分辨率的提升,虽然为地物目标带来更加丰富的细节信息,,但同时也为极化SAR图像解译工作带来了新的阻碍。其次,随着对极化SAR图像特征表征技术的研究发展,涌现了越来越多的极化特征,这些特征从多个不同的角度对地物目标进行描述,如何合理且全面地利用这些极化特征也称为极化SAR图像解译的难题。最后,随着极化SAR技术的不断发展,获取到的极化SAR数据集呈现爆炸性增长趋势,但是标记这些数据集依赖专业知识,需要较大的人工成本和精确的标注技术,由于人工标记错误、自动标记技术精度不足等因素导致的错误标记样本问题给极化SAR图像解译带来巨大的困难\citing{tu2020robust,uhlmann2013integrating,liu2016large}。因此如何利用混杂有噪声标签的有限标记的数据集也成为极化SAR图像解译中的新挑战。具体而言,体现在:

(1)极化SAR图像表现出高分辨率趋势,其中具有更加丰富的地物散射信息。目前,极化SAR特征提取技术发展迅猛,极化特征提取方法日渐增多,这些方法从不同的角度对地物目标的散射特性进行描述。然而,当前大多数的极化SAR图像分类算法仅仅采用了单一类型的极化特征。然而,对于空间环境复杂的场景,具有更复杂的集合结构。单一类型的极化特征无法全面地描述目标的细节信息,无法对目标信息进行全面概括。同时,固定的极化特征表示方式无法在所有类型的地物目标和图像中取得满意的结果。

(2)极化SAR技术不断发展,极化SAR数据集数量呈现爆发式增长。对极化SAR数据集的标注工作依赖专业知识或自动标注技术,由于极化SAR的复杂成像特性,标注工作中不可避免地由于人工错误标记、自动标注技术精度不足等问题出现错误标记的样本。如何高效的利用混杂有噪声样本的数据集是当前极化SAR图像分类研究的难点。

本文旨在利用深度学习技术优势,基于地物目标的散射特性差异,探索极化SAR图像的极化特征最优表征方式。同时利用深度学习技术优势,实现在混杂有噪声样本的有限数据集中的可靠分类,提高极化SAR图像的分类准确率,进而提升极化SAR系统的实际可用性。

\section{国内外相关研究现状}
极化SAR图像分类技术中包含两个关键步骤,即极化特征表示和分类器的设计。本文的研究内容致力于极化特征的表示方法和分类器设计方法。因此,将从极化特征表示和目标分类方法这两个方面来进行现状分析。

\subsection{极化特征表示方法}
极化SAR图像中蕴含了丰富的极化信息,而这些信息的发现和应用都依赖于有效的极化特征表示方法。极化特征表示在极化SAR图像解译任务中扮演着至关重要的基础角色。

像素作为极化SAR图像的基本组成单元,现有的大多数极化SAR图像特征表示和分类方法都是以像素为最小单元进行处理,也就是利用每个像素点本身的特征来完成特征表示和分类。一般而言,常见的极化特征表示方法有对观测数据进行简算数运算构造的特征、基于极化目标分解理论的目标分解特征、图像空间纹理特征等。

(1)对观测数据进行算数运算构造的特征

极化SAR图像的观测数据是指以极化散射矩阵、经过多视处理的相干矩阵和协方差矩阵为代表的相关数据。通过对这些观测数据进行简单的代数运算可以得到用于表征地物部分散射特性的简单特征,主要包括观测数据的相位、幅度以及对应的四则运算等。Rignot等人\citing{rignot1992unsupervised}提出了使用极化协方差矩阵中的各个元素的对数来实现极化SAR图像的分割;Pierce等人\citing{pierce1994knowledge}结合极化散射矩阵和专家系统展开了对极化SAR图像的分类研究;Kong等人\citing{1988Identification}提出了利用极化散射矩阵每个通道的强度之比来进行极化SAR图像分类研究工作;Lee等人\citing{964970}使用散射矩阵计算出相位差,结合最大似然分类器进行分类;Ding等人\citing{1017062722.nh}从散射矩阵和相干矩阵中的各个元素组成特征向量,用于分类研究。基于观测数据简单代数运算的特征简单易得,属于最基本的极化特征,但是缺少了对地物目标的具体物理散射机制的深层解释。

(2)基于极化目标分解理论的目标分解特征

极化目标分解理论通过从物理意义层面对极化SAR数据进行约束,来反映地物目标确切物理意义的散射特性,是极化SAR数据利用中最为经典且至关重要的理论之一。Huynen在1970年首次提出了目标分解理论\citing{huynen1970phenomenological,huynen1990stokes},为后续研究工作奠定了理论基础。极化目标分解按照处理的数据不同,可以分为极化相干分解和极化非相干分解两种方式。极化相干分解的处理对象是散射矩阵,利用不同的基将散射矩阵分解为多个具有不同物理意义的矩阵分量的加权和的形式。极化非相干分解的处理对象是相干矩阵或协方差矩阵,也是将其分解为多个具有不同物理意义的分量的加权和的形式。

极化相干分解针对散射矩阵数据,从物理层面进行约束,利用不同的基向量将散射矩阵分解为不同物理意义的矩阵分量加权和。经典的极化相干分解方法包括Pauli分解\citing{cloude1996review}、Krogager分解\citing{krogager1990new}、Cameron分解\citing{cameron1990feature}等。Puali分解方法是将散射矩阵分解为单次散射、方向角为0°和45°的二次散射的组合;Krogager分解则是将散射矩阵分解为球散射、二面角散射和螺旋体散射的组合形式;Cameron分解基于Pauli分解方法,提出了对定向角旋转之后的最大散射成分作为分解特征的方法。相干分解要求目标具有稳定的散射性质和相干的散射波,适用于没有噪声的环境,对于存在一个主要目标的场景效果优越,能够准确地表达地物目标的物理散射特性。

极化非相干分解方法针对多视矩阵的数据处理,将其分解为多个反映确切物理意义的参数加权和的形式。经典的非相干分解方法包括Cloude分解\citing{cloude1997entropy}、Freeman分解\citing{freeman1998three}、Holm分解\citing{holm1988radar}等。Cloude分解方法利用矩阵的特征值理论,基于矩阵的最大特征值来推断目标的散射类型,还提出了一系列的散射特征,包括散射熵、各项异度性和平均散射角等极化特征;Freeman分解属于基于模型的分解方法,按照地物目标的散射差异,分为平面散射、二次散射和体散射三种类型,其中平面散射主要出现在表面光滑的地物区域,二次散射主要出现在城市区域,尤其是密集建筑物群的地域,体散射主要发生在森林、灌木从等植被茂密区域。

目标分解理论为极化SAR数据处理提供了一系列具有特定物理含义的极化特征,从不同的角度反映的地物目标的物理散射特性,是当前极化SAR特征提取领域最为重视、应用最为广泛的方法。一般而言,通过一种目标分解方法,可以得到1至4个对应的特征参数,而不同的目标分解方法能从不同的角度反映地物的散射机制和地物类型。在某些复杂的地形场景中,不同的目标可能会产生不同的散射回波,相同类型地物目标也有可能产生不同的回波\citing{tu2011laplacian}。因此,采用单一的极化特征描述方法往往无法获得理想的结果。

(3)图像空间纹理特征

在极化SAR图像分类领域,空间纹理特征作为极化特征的补充表示,是对地物目标空间细节特性描述的重要特性之一。直方图统计、Gabor滤波、小波变换等空间纹理特征提取方法在图像解译领域发挥了重要作用。Haralick等人\citing{haralick1973textural}基于灰度共生概率的纹理特征来对SAR图像分类,并通过实验证实了方案可行性;Fauvel等人\citing{fauvel2008spectral}通过基于形态学的高光谱高分辨城市图像分类方法与支持向量机(Support Vector Machine,SVM)相结合的形式,通过以上形态学方法,有效的提升了分类准确率;He等人\citing{he2013texture}结合小波信息与稀疏编码进行特征提取,以SVM作为分类器完成极化SAR图像分类;Ji等人\citing{ji2015segmentation}则结合局部二值模式和HSV空间的颜色特征完成极化SAR图像分类任务。尽管空间纹理特征作为一种易于实现的补充特征在极化SAR图像特征提取领域取得了良好的表现,但是提取范围通常收到尺度限制,无法自适应的对尺度进行缩放而导致分类精度受到限制。


\subsection{极化SAR目标分类方法}
极化SAR图像分类方法作为极化SAR图像解译的关键步骤之一\citing{bi2018graph},受到了国内外诸多研究学者的关注和重视。极化SAR图像分类任务本质是将图像中的每一个像素规矩分类规则赋予一个类别标签,确定所属的地物类别。图\ref{fig:极化SAR分类流程}展示了极化SAR图像数据处理以及目标分类方法的基本流程,整个流程包括极化SAR数据处理、极化特征表示、样本预处理、分类方法四个步骤。其中,目标分类方法是通过对输入极化SAR数据样本的学习,形成按照样本特征进行分类的规则,根据分类规则对剩下的测试样本完成目标分类的流程。当前的极化SAR图像分类算法根据训练过程中使用人工标记样本的数量情况可以分为无监督分类、有监督分类和半监督分类三种类型。

\begin{figure}[h]
    \includegraphics[width=12.3cm]{pic/chapter1/极化SAR分类流程.png}
    \caption{极化SAR分类流程}
    \label{fig:极化SAR分类流程}
\end{figure}


\subsubsection{无监督分类方法}
无监督学习分类方法是指在缺失地物标签信息的情况下完成对目标的分类任务。极化SAR无监督分类任务中,主要是依据极化SAR数据的内在不同特征分布情况,来区分不同类型的地物目标,并且最终每个类别的标签需要人工进行标注。Van Zyl等人\citing{van1989unsupervised}提出了基于入射、反射波之间关系的分类准则,并且首次将极化SAR的目标散射特性运用到目标分类方法中;Cloude等人\citing{cloude1997entropy}创新性地提出了极化熵$\mathrm{H}$以及平均散射角$\bar{\alpha}$的极化特征描述概念,并在$\mathrm{H}-\bar{\alpha}$平面中划分了8个代表不同散射机理的区域,来表示不同的地物类型;Lee等人\citing{lee1999unsupervised}在Cloude的研究基础上进一步提出了利用复Wishart分布的无监督学习方法,开创新地将地物目标物理散射特性和统计特征相结合,为后续的无监督研究工作提供了启发性的思路。Ferro-Famil等人\citing{ferro2001unsupervised}引入了在Cloude的研究基础上引入了各项异度性$\mathrm{A}$,并且进一步地将$\mathrm{H}-\bar{\alpha}$的8区域扩展为16个区域,使分类结果更加准确,包含更多的细节;Yamaguchi等人\citing{kimura2003pi}将散射总功率SPAN与分类方法相结合,获得更好收敛性的无监督分类方法;Lee等人\citing{lee2004unsupervised}在Freeman分解的基础上,将基于复Wishart分类方法与Freeman分解相结合,通过先进行模糊分类,然后进行精细分类的方式,最后利用Wishart进行迭代分类得到最终的分类结果;Bi等人\citing{bi2017polsar}提出了一种基于目标分解理论和K-Wishart似然分类方法相结合的无监督分类方法,基于Pottier和Lee的方法先产生粗分类结果,然后利用多个不同的极化特征和K-Wishart分类方法的最大似然算法实现精确分类。尽管无监督分类方法充分利用了极化SAR数据的目标散射特性和统计分布特性,但是由于标记样本的缺失,同物异谱和异物同谱的情况下往往分类效果较差。

\subsubsection{有监督分类方法}
相比于无监督分类方法,有监督分类方法在确定分类规则时能够使用有标签的数据来进行匹配训练,通常表现出更加优越的分类性能。基于统计理论的分类方法是极化SAR图像分类领域应用最为广泛的方法,也被成为基于贝叶斯理论的分类方法。Kong等人\citing{1988Identification}从极化矢量服从高斯分布这一特性出发,构建了基于复高斯分布的最大似然分类器;Lee等人\citing{lee1999unsupervised,lee1994classification}利用傅相干矩阵服从Wishart分布理论基础,提出了Wishart分类器。

除了以上提及的基于极化SAR数据分布特性的有监督分类方法,近年来,由于深度学习的自动挖掘数据鉴别特征和优越的分类性能,广泛的应用于自然语言处理、智能驾驶、信息检索等领域。由于其优越的性质,越来越多的研究将深度学习方法应用到遥感图像解译领域中,也获得了巨大的成果。Lv等人\citing{lv2014classification}将深度信念网络(Deep Belief Network, DBN)应用到极化SAR图像分类中,并取得了优异的分类效果;Xie等人\citing{xie2014multilayer}利用堆叠式自编码器(Stacked Sparse Autoencoder, SSAE)来获取可鉴别特征,随后基于标签微调,实现了分类性能和区域分类视觉效果上的巨大提升。Zhou等人\citing{zhou2016polarimetric}使用一维向量来表征极化SAR数据的相干矩阵,并将其作为CNN的输入作为分类,开创性地将卷积网络分类方法引入到极化SAR图像分类中。Zhang等人\citing{zhang2017complex}在Zhou的研究基础上提出了基于复数的卷积网络结构(Complex-valued CNN, CV-CNN),通过将卷积网络中的基本模块,包括卷积层、池化层、激活函数、全连接层等均扩展至复数的形式,并且提出了适用于复数域的反向传播优化算法,进一步提升了CNN在极化SAR图像分类任务中的分类准确率,并且为后续的一系列复数域分类算法奠定了基础。现有的基于深度学习的极化SAR图像分类算法中,大多数是将极化SAR观测数据的散射矩阵或相干矩阵直接作为网络的输入,而没有充分利用现有的多种类型的极化特征,并且提取的特征缺乏物理意义。因此,如何利用深度网络来有效探索极化SAR的可鉴别特征仍然需要进一步的研究。

\subsubsection{半监督分类方法}
半监督的分类学习方法是同时利用有标签样本数据和无标签样本数据对分类模型进行训练的方法,属于有监督学习方法和无监督学习方法的有效融合。实际上,在遥感数据中,尤其是SAR数据,要获得大量的标记样本往往是非常困难的,有标记样本数量有限,而无标记的样本大量存在,并且包含了极化SAR数据分布的特征。因此,如何利用大量的无标签数据和少量的标记样本来辅助网络学习到更加准确的分类规则,已经成为一个研究重点。Liu等人\citing{liu2016large}通过建图的方法,利用选定的锚点来构建邻接矩阵,实现从有标签样本到无标签样本的标签传播;Hou等人\citing{hou2017robust}基于字典学习和区域一致特性,学习到稀疏的高级特征之后结合无标签样本实现分类器的训练;Jie等人\citing{geng2017semisupervised}基于超像素方法来应对极化SAR中存在的相干斑噪声的问题;Liu等人\citing{liu2018fully}和Liu等人\citing{liu2019task}利用无标签样本数据,通过生成对抗网络学习到数据的分布特性,实现生成样本扩充训练集数量。

\section{论文主要内容及章节安排}
针对极化SAR分辨率提升、噪声标签必然存在的情况下的极化SAR图像分类性能提升问题,本文开展了基于极化SAR图像特征表示和目标分类的相关研究,重点研究关于极化特征融合。本文共分为五个章节,各章节结构安排为:

第一章:绪论。阐述本文工作的研究背景与意义,并且对其中的极化特征提取、极化SAR目标分类方法进行国内外研究现状进行概括分析,最后给出全文的结构安排。

第二章:极化SAR领域理论基础。首先介绍了电磁波与极化的描述方法,随后对极化SAR的数据表征方式进行了详细介绍,包括散射数据表征方式和几种经典的目标分解理论。

第三章:基于双通道注意力的极化信息提取方法。

第四章:

第五章:对本文研究工作进行总结,并展望未来工作。
\chapter{极化SAR与深度学习相关理论基础}
\section{引言}
极化SAR技术通过发送和接收不同极化方式的电磁波,提供丰富的目标极化散射信息,具有全天时、全天候成像特点。因此,近年来,极化SAR在城市规划、军事探测、灾害检测、农作物监控等多个领域有着广泛的应用\citing{pramudya2019estimation,dumitru2018sar,liu2019small}。极化SAR数据中蕴含丰富的极化信息,如何挖掘这些极化信息,是极化SAR目标分类任务的关键步骤。同时,深度学习技术以其在遥感数据处理中的卓越表现,为极化SAR目标分类任务提供了新的思路和解决方法\citing{liu2016pol, liu2019task, bi2018graph}。本章将介绍极化SAR与深度学习方法的相关理论基础,主要包括目标散射机理介绍,极化数据的表征形式和几种典型的目标分解方法,同时还介绍了深度学习方法理论。

\section{极化电磁波表征方式}
电场与磁场的相互作用产生电磁波。在电磁波传播过程中,电场与磁场振荡平面始终保持垂直,并且都与电磁波的传播方向相垂直,如图\ref{电磁波传输过程示意图}所示:

\begin{figure}[h]
    \includegraphics[width=10.3cm]{pic/chapter2/电磁波传输过程.pdf}
    \caption{电磁波传输过程示意图}
    \label{电磁波传输过程示意图}
\end{figure}

极化是电磁波的基本特性之一,是指在固定的空间点处,电场振荡方向随着时间的变化方式。在笛卡尔坐标系中,设定电磁波的传播方向为$\hat{z}$轴正方向,电场与磁场的关系可以使用麦克斯韦方程组表示\citing{皮亦鸣2007合成孔径雷达成像原理, 路宏敏2006电磁场与电磁波基础}:

\begin{gather}
    \label{Maxwell1}
    \nabla \times \boldsymbol{H}(z,t)=\epsilon \frac{\partial \boldsymbol{E}(z,t)}{\partial t} \\
    \nabla \times \boldsymbol{E}(z,t)=-\mu \frac{\partial \boldsymbol{H}(z,t)}{\partial t}     \\
    \nabla \cdot \boldsymbol{H}(z,t)=0                                                         \\
    \label{Maxwell4}
    \nabla \cdot \boldsymbol{E}(z,t)=0
\end{gather}
对公式\ref{Maxwell1}等号左右两边取旋度,有:
\begin{equation}
    \nabla \times(\nabla \times \boldsymbol{E})=-\mu \frac{\partial}{\partial t}(\nabla \times \boldsymbol{H})
\end{equation}
通过矢量恒等式$\nabla \times(\nabla \times \boldsymbol{E})=\nabla(\nabla \cdot \boldsymbol{E})-\nabla^2 \boldsymbol{E}$和公式\ref{Maxwell4}可以得出电磁波的波动方程:
\begin{equation}
    \nabla^2 \boldsymbol{E}(z, t)-\mu \epsilon \frac{\partial^2 \boldsymbol{E}(z, t)}{\partial t^2}=0
\end{equation}
SAR系统与地面目标之间的距离满足远场条件,可以将SAR工作过程中收发的电磁波近似为平面波。通过波动方程,可以得到平面波的表达式:
\begin{equation}
    \label{平面波}
    \boldsymbol{E}(z, t)=\left[\begin{array}{c}
            E_x(z, t) \\
            E_y(z, t) \\
            0
        \end{array}\right]=\left[\begin{array}{c}
            E_{0 x} \cos \left(\omega t-k z+\delta_x\right) \\
            E_{0 y} \cos \left(\omega t-k z+\delta_y\right) \\
            0
        \end{array}\right]
\end{equation}
其中,$E_{0x}$, $E_{0y}$, $\delta_x$ 和 $\delta_y$分别表示电场在$x$与$y$方向上的幅度和初始相位,$k$ 表示传播常量。
\subsection{极化椭圆}
在空间中某个固定的点处电场矢量端点的轨迹可以用来描述电磁波的极化状态。对于某个时刻$t$,由公式\ref{平面波}可得:
\begin{equation}
    \label{轨迹方程}
    \left(\frac{E_x(z, t)}{E_{x_0}}\right)^2-2\left(\frac{E_x(z, t) E_y(z, t)}{E_{x_0} E_{y_0}}\right) \cos (\delta)+\left(\frac{E_y(z, t)}{E_{y_0}}\right)^2=\sin ^2(\delta)
\end{equation}
其中,$\delta=\delta_x-\delta_y$,$-\pi \leqslant \delta \leqslant \pi$。

\begin{figure}[h]
    \includegraphics[width=7.3cm]{pic/chapter2/极化椭圆.jpg}
    \caption{极化椭圆示意图\citing{皮亦鸣2007合成孔径雷达成像原理}}
    \label{极化椭圆示意图}
\end{figure}

由公式\ref{轨迹方程}在$z_0$处的运动轨迹可以得到极化椭圆,如图\ref{极化椭圆示意图}所示。对于任意一个极化椭圆。可以通过椭圆幅度$A$,椭圆方向角$\chi \in [-\frac{\pi}{4}, \frac{\pi}{4}]$和椭圆孔径$\varphi \in [-\frac{\pi}{2}, \frac{\pi}{2}]$进行唯一表示:
\begin{gather}
    \mathbf{A}=\sqrt{\left|E_{x_0}\right|^2+\left|E_{y_0}\right|^2}                                                                \\
    \tan 2 \varphi=\frac{2\left|E_{x_0}\right|\left|E_{y_0}\right|^2}{\left|E_{x_0}\right|^2-\left|E_{y_0}\right|^2} \cos (\delta) \\
    \sin 2 \chi=\frac{2\left|E_{x_0}\right|\left|E_{y_0}\right|^2}{\left|E_{x_0}\right|^2+\left|E_{y_0}\right|^2} \sin (\delta)
\end{gather}
根据电场矢量的旋转方向,可以将极化电磁波分为左旋和右旋两种类型。当椭圆方向角$\chi > 0$时,表示左旋极化波;当椭圆方向角$\chi < 0$时,表示左旋极化波。
\subsection{琼斯矢量}
为了简化平面电磁波极化状态的描述方式,还可以通过琼斯矢量的形式表示。将公式\ref{平面波}使用复数的形式表示为:
\begin{equation}
    \boldsymbol{E}(z, t)=\left[\begin{array}{c}
            E_{0 x} \cos \left(\omega t-k z+\delta_x\right) \\
            E_{0 y} \cos \left(\omega t-k z+\delta_y\right)
        \end{array}\right]=\operatorname{Re}\left\{\left[\begin{array}{l}
            E_{0 x} e^{j \delta_x} \\
            E_{0 y} e^{j \delta_y}
        \end{array}\right] e^{j \omega t} e^{-j k z}\right\}
\end{equation}
其中,$Re(\cdot)$表示取实部。琼斯矢量可以定义为:
\begin{equation}
    E_{\text {Jones }}=\left[\begin{array}{l}
            E_x \\
            E_y
        \end{array}\right]=\left[\begin{array}{l}
            E_{0 x} e^{j \delta_x} \\
            E_{0 y} e^{j \delta_y}
        \end{array}\right]
\end{equation}
极化椭圆描述的极化波状态与琼斯矢量表示是等价的,可以使用椭圆极化的描述参数来表示琼斯矢量:
\begin{equation}
    E_{\text {Jones }}=\mathbf{A} e^{i a}\left[\begin{array}{l}
            \cos \varphi \cos \chi-i \sin \varphi \sin \chi \\
            \sin \varphi \cos \chi+i \cos \varphi \sin \chi
        \end{array}\right]
\end{equation}
其中,$\alpha$表示绝对相位项。琼斯矢量可以通过更加简洁的矩阵形式表示,定义为:
\begin{equation}
    E_{\text {Jones }}=\mathbf{A} e^{i a}\left[\begin{array}{cc}
            \cos \varphi & -\sin \varphi \\
            \sin \varphi & \cos \varphi
        \end{array}\right]\left[\begin{array}{c}
            \cos \chi \\
            i \sin \chi
        \end{array}\right]
\end{equation}
\subsection{Stokes矢量}
与琼斯矢量使用两个复数表示电磁波的极化状态不同,Stokes矢量使用$[g_0,g_1,g_2,g_3]$四个实数从电磁波功率的角度来描述电磁波的极化状态,Stokes参数定义如下:
\begin{equation}
    \label{Stokes}
    \left\{\begin{array}{l}
        g_0=\left|E_x\right|^2+\left|E_y\right|^2         \\
        g_1=\left|E_x\right|^2-\left|E_y\right|^2         \\
        g_2=2\left|E_x\right|\left|E_y\right| \cos \delta \\
        g_3=2\left|E_x\right|\left|E_y\right| \sin \delta
    \end{array}\right.
\end{equation}
其中,$E_x$和$E_y$分别表示电场矢量在x轴和y轴上的幅度;$\delta$表示电场矢量在x轴和y轴上的相位差;$g_0$表示电磁波总功率;$g_1$表示水平或垂直极化分量的功率;$g_2$表示$\pm 45^{\operatorname{\omicron}}$线性极化分量的功率;$g_3$表示左右旋极化分类的功率之和。

可以使用极化椭圆参数表示公式\ref{Stokes},具体如下:
\begin{equation}
    \boldsymbol{g}=\left[\begin{array}{c}
            A^2                             \\
            A^2 \cos (2 \phi) \cos (2 \tau) \\
            A^2 \sin (2 \phi) \cos (2 \tau) \\
            A^2 \sin (2 \tau)
        \end{array}\right]
\end{equation}

\section{极化散射数据表征方式}
极化SAR系统通过发射水平和垂直两种不同极化方式的电磁波,对目标进行探测,当电磁波遇到目标后,部分被目标体吸收,另外一部分通过目标辐射,形成反射回波。由于散射回波因目标的散射特性而异,因此可以通过分析回波的特性来推断目标的特征。为了能够表征目标的极化散射特性,需要引入不同的参数来描述各个极化通道散射回波在幅度相位上的差异。
\subsection{极化散射矩阵和Mueller矩阵}
极化SAR系统从发射的不同极化形式的电磁波信号,到以不同的极化方式接收经过目标反射的电磁波,目标的反射过程可以看做电磁波对应琼斯矢量的线性转换的过程。为了描述这个线性转换的过程,引入极化散射矩阵\citing{sinclair1950transmission},又称为Sinclair矩阵,简写为$S$,通常使用$2\times2$的复数矩阵表示,具体如下:
\begin{equation}
    \left[ S \right] =\left[ \begin{matrix}
            S_{HH} & S_{HV} \\
            S_{VH} & S_{VV} \\
        \end{matrix} \right]
\end{equation}
其中,$H$和$V$分别表示水平和垂直的极化方式,$S_{XY}(X,Y=H,V)$表示发射$X$极化波、接收$Y$极化波的后向散射系数,在满足单站互易条件下,$S_{HV}=S_{VH}$。此时,极化散射矩阵可以表示为如下形式:
\begin{equation}
    \left[ S \right] =\left[ \begin{matrix}
            S_{HH} & S_{HV} \\
            S_{HV} & S_{VV} \\
        \end{matrix} \right]
\end{equation}

以上的极化散射矩阵适用于表述完全极化波对应的琼斯矢量的入射与散射之间的线性关系,但是通常情况下,入射与散射的电磁波都是部分极化电磁波,这种情况下,需要引入Mueller矩阵\citing{guissard1994mueller}来进行表示它们之间的关系。

Mueller矩阵是一个$4\times4$的矩阵,简写为$M$,具体表示如下:
\begin{equation}
    \label{Mueller}
    M=R\left( S\otimes S^* \right) R^{-1}
\end{equation}
其中,$S$是Sinclair矩阵,$R$是转换系数矩阵,具体如下:
\begin{equation}
    \label{R}
    R=\left[ \begin{matrix}
            1 & 0 & 0  & 1  \\
            1 & 0 & 0  & -1 \\
            0 & 1 & 1  & 0  \\
            0 & i & -i & 0  \\
        \end{matrix} \right]
\end{equation}

从公式\ref{Mueller}和公式\ref{R}可以看出,$M$实际上是经过$S$运算得到,因此$M$与$S$属于一一对应的关系,但是$S$相比于$M$还多具备了绝对的相位信息。

\subsection{极化相干矩阵和协方差矩阵}
在极化SAR系统成像过程中,目标的散射回波中往往还夹杂了周围其他散射体的散射杂波。为了减少散射杂波的影响,在实际应用过程中通常会对极化散射矩阵$S$进行矢量化分解,得到二阶统计的极化相干矩阵和极化协方差矩阵。

通常利用正交的矩阵基来对极化散射矩阵进行矢量化分解,矢量化过程表示如下:
\begin{equation}
    S=\left[ \begin{matrix}
            S_{HH} & S_{HV} \\
            S_{HV} & S_{VV} \\
        \end{matrix} \right] \Rightarrow K_4=V(S)=\frac{1}{2}\mathrm{Trace(}S\Psi )=\left[ k_0,k_1,k_2,k_3 \right] ^T
\end{equation}
其中,$V(\cdot)$表示矢量化操作,$Trace(\cdot)$表示矩阵求迹,$\Psi$表示$2\times2$的复单位矩阵,$T$表示矩阵转置。最常用的有两组标准基,分别是Lexicograhic基$Psi_L$和Pauli基$\Psi_p$,具体如下:
\begin{gather}
    \Psi_L=\left\{\left[\begin{array}{ll}
            2 & 0 \\
            0 & 0
        \end{array}\right],\left[\begin{array}{ll}
            0 & 2 \\
            0 & 0
        \end{array}\right],\left[\begin{array}{ll}
            0 & 0 \\
            2 & 0
        \end{array}\right],\left[\begin{array}{ll}
            0 & 0 \\
            0 & 2
        \end{array}\right]\right\}                                    \\
    \Psi_P=\left\{\sqrt{2}\left[\begin{array}{ll}
            1 & 0 \\
            0 & 1
        \end{array}\right], \sqrt{2}\left[\begin{array}{cc}
            1 & 0  \\
            0 & -1
        \end{array}\right], \sqrt{2}\left[\begin{array}{ll}
            0 & 1 \\
            1 & 0
        \end{array}\right], \sqrt{2}\left[\begin{array}{cc}
            0 & -i \\
            i & 0
        \end{array}\right]\right\}
\end{gather}

基于以上两组基,分别得到目标散射矢量:
\begin{gather}
    \label{4L}
    k_{4 L}=\left[S_{HH}, S_{HV}, S_{VH}, S_{VV}\right] \\
    \label{4P}
    k_{4 P}=\frac{1}{\sqrt{2}}\left[S_{HH}+S_{VV}, S_{HH}-S_{VV}, S_{HV}+S_{VH}, i\left(S_{HV}+S_{VH}\right)\right]
\end{gather}

由单站互易定理可知$S_{HV}=S_{VH}$,可以将公式\ref{4L}和\ref{4P}简写为:
\begin{equation}
    \begin{gathered}
        k_{3 L}=\left[S_{HH}, \sqrt{2}S_{HV}, S_{VV}\right] \\
        k_{3 P}=\frac{1}{\sqrt{2}}\left[S_{HH}+S_{VV}, S_{HH}-S_{VV}, 2S_{HV}\right]
    \end{gathered}
\end{equation}

通过计算散射矢量$k_{3L}$与其复共轭转置矢量$k_{3L}^{H}$的内积可以得到极化协方差矩阵$C$,具体表示如下:
\begin{equation}
    \boldsymbol{C}=\left. \langle k_{3L}\cdot k_{3L}^{^*\boldsymbol{T}} \right. \rangle =\left[ \begin{matrix}
            \left. \langle \left. S_{HH} \right|^2 \right. \rangle  & \left. \langle \sqrt{2}S_{HH}S_{HV}^{*} \right. \rangle  & \left. \langle S_{HH}S_{VV}^{*} \right. \rangle         \\
            \left. \langle \sqrt{2}S_{HV}S_{HH}^{*} \right. \rangle & \left. \langle 2\left| S_{HV} \right|^2 \right. \rangle  & \left. \langle \sqrt{2}S_{HV}S_{VV}^{*} \right. \rangle \\
            \left. \langle S_{VV},S_{HH}^{*} \right. \rangle        & \left. \langle \sqrt{2}S_{VV},S_{HV}^{*} \right. \rangle & \left. \langle \left| S_{VV} \right|^2 \right. \rangle  \\
        \end{matrix} \right]
\end{equation}

同理,通过计算散射矢量$k_{3P}$与其复共轭转置矢量$k_{3P}^{H}$的内积可以得到极化相干矩阵$T$,具体表示如下:
\begin{equation}
    \label{eq:T}
    \begin{aligned}
        T & =\left. \langle k_{3L}\cdot k_{3L}^{^*\boldsymbol{T}} \right. \rangle                                                                                                                                                                                                                  \\
          & =\frac{1}{2}\left[ \begin{matrix}
                                       \langle \left| S_{HH}+S_{VV} \right|^2\rangle                                               & \left. \langle \left( S_{HH}+S_{VV} \right) \left( S_{HH}-S_{VV} \right) ^* \right. \rangle & \left. \langle 2\left( S_{HH}+S_{VV} \right) S_{HV}^{*} \right. \rangle \\
                                       \left. \langle \left( S_{HH}-S_{VV} \right) \left( S_{HH}+S_{VV} \right) ^* \right. \rangle & \left. \langle \left| S_{HH}-S_{VV} \right|^2 \right. \rangle                               & \left. \langle 2\left( S_{HH}-S_{VV} \right) S_{HV}^{*} \right. \rangle \\
                                       \left. \langle 2S_{HV}\left( S_{HH}+S_{VV} \right) ^* \right. \rangle                       & \left. \langle 2S_{HV}\left( S_{HH}-S_{VV} \right) ^* \right. \rangle                       & \left. \langle 4\left| S_{HV} \right|^2 \right. \rangle                 \\
                                   \end{matrix} \right]
    \end{aligned}
\end{equation}
其中,$\langle \cdot \rangle$表示取总体平均,$(\cdot)^*$和$(\cdot)^T$分别表示复共轭和矩阵转置。
极化协方差矩阵$C$与极化相干矩阵都是半正定的Hermitian矩阵,两者之间可以相互转换:
\begin{gather}
    C=U^H T U=U^{-1} T U \\
    T=U C U^H=U C U^{-1}
\end{gather}
其中,$U$表示单位转换矩阵,具体可以表示为:
\begin{equation}
    U=\frac{1}{\sqrt{2}}\left[ \begin{matrix}
            1 & 0        & 1  \\
            1 & 0        & -1 \\
            0 & \sqrt{2} & 0  \\
        \end{matrix} \right]
\end{equation}

\section{极化目标分解理论基础}
极化散射矩阵全面的描述了目标的散射特征,可以从中获取到目标表面粗糙度、对称性以及取向性等信息,为深层次地探索目标特性提供了数据支撑。但是,仅仅基于测量到的散射数据并无法直接获取丰富的信息,还需要进一步对测量的散射矩阵进行数据处理,进而从其中提取出能表征目标特征的极化描述。为了更清晰的描述不同目标的极化特性,可以将散射矩阵通过不同的散射模型作为基分解为几个矩阵的线性组合,由于这些基础散射模型具有不同的物理意义,分解矩阵可以表示目标的不同散射特性,为后续的识别、分类任务提供更多的信息。极化目标分解根据分解的数据分为相干分解和非相干分解两种类别:相干分解的分对象是极化散射矩阵,要求目标具有稳定的散射性质和相干的散射波;非相干分解的分解对象是极化相干矩阵或极化协方差矩阵,要求目标具有非相干散射波,散射的特征随时间变化,探测目标可以是分布式目标。
\subsection{极化相干分解}
极化相干分解的基本思想是通过不同的基础散射模型,将测量得到的散射矩阵分解成多个散射机制之和,如下式所示:
\begin{equation}
    S=\sum_{k=1}^N{a_kS_k}
\end{equation}
其中,$S$表示极化散射矩阵,$S_k$表示分解得到的经典目标的散射矩阵,$a_k$表示对应的权值。

经典的极化相干分解方法有Pauli分解\cite{cloude1996review}、Krogager分解\cite{krogager1990new}、Cameron分解\cite{cameron1990feature}、球坐标分解\cite{lee2017polarimetric}等。下面将介绍Pauli分解和Krogager分解的基本原理。
\subsubsection{Pauli分解}
Pauli分解方法是使用Pauli基对极化散射矩阵进行分解,分解过程如下式所示:
\begin{equation}
    S=\left[ \begin{matrix}
            S_{HH} & S_{HV} \\
            S_{VH} & S_{VV} \\
        \end{matrix} \right] =aS_a+bS_b+cS_c+dS_d
\end{equation}
其中,$a,b,c,d$均是权重系数,$S_a,S_b,S_c,S_d$表示Pauli基,具体如下式所示:
\begin{equation}
    S_a=\left[ \begin{matrix}
            1 & 0 \\
            0 & 1 \\
        \end{matrix} \right] ,S_b=\left[ \begin{matrix}
            1 & 0  \\
            0 & -1 \\
        \end{matrix} \right] ,S_c=\left[ \begin{matrix}
            0 & 1 \\
            1 & 0 \\
        \end{matrix} \right] ,S_a=\left[ \begin{matrix}
            0 & -i \\
            i & 0  \\
        \end{matrix} \right]
\end{equation}

权重系数使用向量的形式,可以表示为:
\begin{equation}
    K=\left[ \begin{matrix}
            a & b & c & d \\
        \end{matrix} \right] =\frac{1}{\sqrt{2}}\left[ \begin{matrix}
            S_{HH}+S_{VV} & S_{HH}-S_{VV} & S_{HV}+S_{VH} & i\left( S_{VH}-S_{HV} \right) \\
        \end{matrix} \right] ^T
\end{equation}

满足单站互易条件时,可以简化为:
\begin{equation}
    K=\left[ \begin{matrix}
            a & b & c \\
        \end{matrix} \right] =\frac{1}{\sqrt{2}}\left[ \begin{matrix}
            S_{HH}+S_{VV} & S_{HH}-S_{VV} & 2S_{HV} \\
        \end{matrix} \right] ^T
\end{equation}

对于Pauli分解的每个分量的物理解释如表\ref{Pauli table}所示。
\begin{table}[h]
    \caption{Pauli分解}
    \linespread{1.5} % 调整整个表格的行高
    \setlength{\arraycolsep}{10pt} % 调整列之间的空白

    \resizebox{\linewidth}{!}{
        \begin{tabular}{|c|c|c|}
            \hline
            Pauli基                 & 散射类型                & 物理描述                             \\ \hline
            $\left[ \begin{matrix}
                                1 & 0 \\
                                0 & 1 \\
                            \end{matrix} \right] $ & 奇次散射                & 球体、平坦平面或三面角反射器           \\ \hline
            $\left[ \begin{matrix}
                                1 & 0  \\
                                0 & -1 \\
                            \end{matrix} \right] $ & 偶次散射                & 二面角反射器                   \\ \hline
            $\left[ \begin{matrix}
                                0 & 1 \\
                                1 & 0 \\
                            \end{matrix} \right] $ & $\frac{\pi}{4}$偶次散射 & 与水平$\frac{\pi}{4}$倾角的二面角 \\ \hline
        \end{tabular}
    }

    \label{Pauli table}
\end{table}
% \begin{figure}[h]
% 	\includegraphics[width=7.3cm]{pic/chapter2/SF-Puali.jpg}
% 	\caption{Pauli分解伪彩图(R:$ \left | a \right |^2$,G:$ \left | b \right |^2$,B:$ \left | c \right |^2$)}
% 	\label{Pauli示意图}
% \end{figure}

\subsubsection{Krogager分解}
Krogager分解是将极化散射矩阵$S$分解为球散射、二面角散射以及螺旋体散射三个散射分量的加权和,具体如下:
\begin{equation}
    \begin{aligned}
        S_{(H, V)} & =e^{j \phi}\left\{e^{j \phi_S} k_S S_{sphere}+k_D S_{diplane(\theta)}+k_H S_{helix(\theta)}\right\}                                            \\
                   & =e^{j \phi}\left\{e^{j \phi_S} k_S\left[\begin{array}{ll}
                                                                     1 & 0 \\
                                                                     0 & 1
                                                                 \end{array}\right]+k_D\left[\begin{array}{cc}
                                                                                                 \cos 2 \theta & \sin 2 \theta  \\
                                                                                                 \sin 2 \theta & -\cos 2 \theta
                                                                                             \end{array}\right]+k_H e^{ \pm j 2 \theta}\left[\begin{array}{cc}
                                                                                                                                                 1     & \pm j \\
                                                                                                                                                 \pm j & -1
                                                                                                                                             \end{array}\right]\right\}
    \end{aligned}
\end{equation}
其中,$k_s$、$k_D$、$k_H$分别表示各个分量的权值,也被称作能量;$\theta$表示取向角;$\phi$表示散射矩阵的绝对相位。

当电磁波以左旋圆极化方式发射,右旋圆极化接收时,Krogager分解可以表示为:
\begin{equation}
    \begin{aligned}
        S_{(R,L)} & =\left[ \begin{matrix}
                                    S_{RR} & S_{RL} \\
                                    S_{LR} & S_{LL} \\
                                \end{matrix} \right]
        \\
                  & =e^{j\phi}\left\{ e^{j\phi _S}k_S\left[ \begin{matrix}
                                                                    0 & j \\
                                                                    j & 0 \\
                                                                \end{matrix} \right] +k_D\left[ \begin{matrix}
                                                                                                    e^{j2\theta} & j              \\
                                                                                                    j            & -e^{-j2\theta} \\
                                                                                                \end{matrix} \right] +k_H\left[ \begin{matrix}
                                                                                                                                    e^{j2\theta} & 0 \\
                                                                                                                                    0            & 0 \\
                                                                                                                                \end{matrix} \right] \right\}
    \end{aligned}
\end{equation}
其中,Krogager的参数可以表示为:
\begin{equation}
    \label{Krogager参数}
    \begin{matrix}
        k_S=\left| S_{RL} \right|,                                    & \phi =\frac{1}{2}\left( \phi _{RR}+\phi _{LL}-\pi \right)          \\
        \theta =\frac{1}{4}\left( \phi _{RR}-\phi _{LL}+\pi \right) , & \phi _S=\phi _{RL}-\frac{1}{2}\left( \phi _{RR}+\phi _{LL} \right) \\
    \end{matrix}
\end{equation}

从公式\ref{Krogager参数}可以看出,目标的左右旋散射特性可以由$S_{RR}$和$S_{LL}$确定。当目标为左螺旋体时:
\begin{equation}
    \left| S_{RR} \right|\geqslant \left| S_{LL} \right|\Rightarrow \left\{ \begin{array}{c}
        k_{D}^{+}=\left| S_{LL} \right|                       \\
        k_{H}^{+}=\left| S_{RR} \right|-\left| S_{LL} \right| \\
    \end{array} \right.
\end{equation}

类似地,当目标为右螺旋体时:
\begin{equation}
    \left| S_{RR} \right|\leqslant \left| S_{LL} \right|\Rightarrow \left\{ \begin{array}{c}
        k_D=\left| S_{RR} \right|                       \\
        k_H=\left| S_{LL} \right|-\left| S_{RR} \right| \\
    \end{array} \right.
\end{equation}

Krogager分解每个分量的物理解释如表\ref{Krogager table}所示。
\begin{table}[h]
    \caption{Krogager分解}
    \resizebox{\linewidth}{!}{
        \begin{tabular}{|c|c|c|}
            \hline
            Krogager基                   & 散射类型  & 物理描述                   \\ \hline
            $\left[ \begin{matrix}
                                1 & 0 \\
                                0 & 1 \\
                            \end{matrix} \right] $      & 球面散射  & 球体、平坦平面或三面角反射器 \\
            \hline
            $\left[ \begin{matrix}
                                cos2\theta & sin2\theta  \\
                                sin2\theta & -cos2\theta \\
                            \end{matrix} \right] $ & 二面角散射 & 二面角反射器              \\ \hline
            $\left[ \begin{matrix}
                                1     & \pm j \\
                                \pm j & -1    \\
                            \end{matrix} \right] $      & 螺旋体散射 & 不对称结构          \\ \hline
        \end{tabular}
    }
    \label{Krogager table}
\end{table}

\subsection{极化非相干分解}
对于“非确定性”的非相干目标,具有动态的散射特性,因此需要使用统计的方法来对其目标散射特性进行研究,通常是使用二阶统计量的方法展开。非相干分解主要是将极化协方差矩阵$C$或者极化相干矩阵$T$进行分解,使用多个典型的散射模型之和的形式表示。
\begin{gather}
    C=\sum_{k=1}^N{p_kC_k}
    \\
    T=\sum_{k=1}^N{q_kT_k}
\end{gather}
其中,$p_k$和$q_k$均表示每个散射模型分量的权重系数。

经典的非相干分解方法包括Cloude分解\citing{cloude1997entropy}、Freeman分解\citing{freeman1998three}、Huynen分解\citing{huynen1988extraction}、Yamaguchi分解\citing{yamaguchi2005four}、Holm分解\citing{holm1988radar}和van Zyl分解\citing{van1993application}等。下面将介绍Cloude分解和Freeman分解的原理。
\subsubsection{Cloude分解}
Cloude分解是一种矩阵特征空间分解方法,是对极化相干矩阵$T$进行特征分解,得到三组特征值与对应的特征向量。具体如下式所示:
\begin{equation}
    \mathrm{T}=\mathrm{U} \Lambda \mathrm{U}^{\mathrm{H}}=\mathrm{U}\left[\begin{array}{ccc}
            \lambda_1 & 0         & 0         \\
            0         & \lambda_2 & 0         \\
            0         & 0         & \lambda_3
        \end{array}\right] \mathrm{U}^{\mathrm{H}}
\end{equation}
其中,$\Lambda$表示矩阵$\mathrm{T}$的三个特征值组成的对角矩阵,并且满足$\lambda_1 \geqslant \lambda_2 \geqslant \lambda_3 \geqslant 0$,$\mathrm{U}$表示三个特征值对应的特征向量构成的矩阵,$\mathrm{H}$表示取复共轭转置。

由以上的特征值与特征向量,Cloude分解定义了平均散射角$\bar{a}$、极化熵$H$以及极化反熵$A$这三个用于描述目标统计的散射机制的量,具体如下表示:
\begin{gather}
    \bar{\alpha}=\sum_{i=1}^3{P_i\alpha _i}
    \\
    H=\sum_{i=1}^3{P_i\log _3P_i}
    \\
    A=\frac{\lambda _2-\lambda _3}{\lambda _2+\lambda _3}
\end{gather}
其中,$\alpha_i$表示从特征值$\lambda_i$中得到的散射角,$P_i=\frac{\lambda _i}{\sum_j{\lambda _j}}$。

从Cloude分解的三个散射分类计算式可以看出,平均散射角$\bar{\alpha}$描述了散射角从$0^{\operatorname{\omicron}}$到$90^{\operatorname{\omicron}}$变化过程中,散射特性从单次散射到二面角散射的连续变化的过程;极化熵$H$描述了目标散射特性的随机性,当$H=1$时,目标是一个随机噪声,当$H=0$时,目标的散射特性唯一,相当于一个相干目标;极化反熵$A$是对极化熵的补充量,通常用来作为极化熵难以区分目标的指标量。

Cloude分解每个分量的物理解释如表\ref{Cloude table}所示。
\begin{table}[h]
    \caption{Cloude分解}
    \resizebox{\linewidth}{!}{
        \begin{tabularx}{\textwidth}{|l|X|X|}
            \hline
            Cloude参数         & 参数表达                                                    & 物理描述                           \\ \hline
            Entropy          & $H=\sum_{i=1}^3{P_i\log _3P_i}$                         & 散射目标由各向性散射至完全随机散射的随机性          \\ \hline
            Mean Alpha Angle & $\bar{\alpha}=\sum_{i=1}^3{P_i\alpha _i}$               & 由表面散射至二面角散射的平均随机性              \\ \hline
            Anisotropy       & $A=\frac{\lambda _2-\lambda _3}{\lambda _2+\lambda _3}$ & 描述了主散射体之外的其余两个相对较弱的散射分量之间的强弱关系 \\ \hline
        \end{tabularx}
    }
    \label{Cloude table}
\end{table}

\subsubsection{Freeman分解}
Freeman和Durden基于van Zyl的目标分解工作,进一步提出了一种非相干的三分量分解方法,将极化协方差矩阵$C$使用单次散射、偶次散射和体散射三种散射机制的加权和表示。具体表示如下:
\begin{equation}
    C=f_sC_s+f_dC_d+f_vC_v
\end{equation}
其中,
\begin{equation}
    \begin{matrix}
        C_s=\left[ \begin{matrix}
                           \left| \beta \right|^2 & 0 & \beta \\
                           0                      & 0 & 0     \\
                           \beta ^*               & 0 & 1     \\
                       \end{matrix} \right] , & C_d=\left[ \begin{matrix}
                                                               \left| \alpha \right|^2 & 0 & \alpha \\
                                                               0                       & 0 & 0      \\
                                                               \alpha ^*               & 0 & 1      \\
                                                           \end{matrix} \right] , & C_v=\left[ \begin{matrix}
                                                                                                   1   & 0   & 1/3 \\
                                                                                                   0   & 2/3 & 0   \\
                                                                                                   1/3 & 0   & 1   \\
                                                                                               \end{matrix} \right] \\
    \end{matrix}
\end{equation}
其中$\beta$ 表示单次散射参数,$\alpha$ 表示偶次散射参数;$C_s$、$C_d$、$C_v$分别表示单次散射模型、偶次散射模型和体散射模型的协方差矩阵分量;$f_s$、$f_d$,$f_v$表示三种散射模型对应的分解系数。

Freeman分解每个分量的物理解释如表\ref{Freeman table}所示。
\begin{table}[!ht]
    \caption{Freeman分解}
    \setlength{\tabcolsep}{10mm}{
        \resizebox{\linewidth}{!}{
            \begin{tabular}{|c|c|c|}
                \hline
                Freeman基                                   & 散射类型 & 典型散射体 \\ \hline
                $\left[ \begin{matrix}
                                    \left| \beta \right|^2 & 0 & \beta \\
                                    0                      & 0 & 0     \\
                                    \beta ^*               & 0 & 1     \\
                                \end{matrix} \right]$      & 单次散射 & 一阶Bragg表面散射体  \\ \hline
                $\left[ \begin{matrix}
                                    \left| \alpha \right|^2 & 0 & \alpha \\
                                    0                       & 0 & 0      \\
                                    \alpha ^*               & 0 & 1      \\
                                \end{matrix} \right]$    & 偶次散射 & 二面角散射器          \\ \hline
                $\left[ \begin{matrix}
                                    1           & 0           & \frac{1}{3} \\
                                    0           & \frac{2}{3} & 0           \\
                                    \frac{1}{3} & 0           & 1           \\
                                \end{matrix} \right]$ & 体散射  & 植被冠状层偶极子           \\ \hline
            \end{tabular}
        }
    }
    \label{Freeman table}
\end{table}

\section{本章小结}
本章详细介绍了极化电磁波及极化散射数据的相关理论基础。首先介绍了电磁波极化的三种表征方式包括极化椭圆、表征完全极化波的琼斯矢量和表征部分极化波的Stokes矢量。然后,介绍了极化SAR数据的散射特性描述形式,包括几种不同的极化矩阵。最后,详细介绍了极化目标分解的方法以及部分方法的详细理论基础,包括相干目标分解和非相干目标分解的代表方法以及每个参数的具体物理意义,还给出了经典方法的极化分解结果,为后续的极化SAR目标分类任务奠定理论基础。
\chapter{基于双通道注意力的极化信息提取方法}
\section{引言}
在上一章中,介绍了目标的散射信息可通过极化散射矩阵$S$进行表示,并通过对$S$进行矢量化分解,可以得到极化信息的进一步表征,包括极化相干矩阵$T$和极化协方差矩阵$C$。使用不同的散射基对$T$或$S$进行分解,得到具有不同物理含义的散射分量。在当前的极化SAR目标分类任务中,基于卷积神经网络(CNN)的分类器通常直接采用协方差矩阵和相干矩阵作为输入,而忽略了极化目标分解的特征表示方式\citing{}。在使用CNN进行分类时,由于其分层特征提取特性,前端网络提取的特征层次相对较低,可能未能充分提取极化SAR数据中的有用信息。同时,如果直接堆叠使用所有极化特征作为输入可能导致特征维度的大幅增加,并且多维特征之间必然存在着信息冗余,可能降低分类准确性\citing{}。

% 针对上述问题,本章研究了一种基于双通道注意力机制的极化信息提取方法。该方法以注意力机制为基础,设计双通道的联合注意力结构,联合不同层次的极化特征进行建模,通过两个单独注意力通道提取不同层次的极化特征并相互校正,最终得到可鉴别极化特征。

针对上述问题,本章提出了一种基于注意力机制的极化信息提取方法。鉴于不同的散射目标之间的散射特性存在差异,引入了关联注意力机制,联合不同层次的极化特征的关系进行联合建模。首先,为了提取不同层次的极化信息,设计了双通道注意力的极化信息提取网络结构,分别对应目标分解通道和散射数据通道。然后,在两个通道中分别利用空间、通道注意力机制,捕捉各自的关键信息。为了挖掘不同层次极化信息的联系,设计了权重融合模块,用于相互引导修正两个通道的注意力权重。最后,结合跨空间学习模块,对不同尺度的极化特征进行再次融合,得到有效的极化特征表示,结合后续的卷积网络与分类器完成目标分类任务。本章提出的双通道注意力的极化信息提取方法,可以作为一个即插即用的插件式组件,应用到现有的任意一个极化SAR目标分类网络中,增强极化信息的表征能力,提高最终的分类准确率。

\section{注意力机制介绍}
注意力机制是机器学习和深度学习中一种关键的技术,其主要目标是在处理信息是实现对输入数据的加权关注,以便网络模型能够更有效地捕捉与任务相关的信息。基于注意力机制的信息提取在自然语言处理、计算机视觉等领域获得了广泛的应用\citing{}。注意力机制使用不同的权重来表示输入特征的不同的重要程度,根据关注的角度差异,可以分为通道注意力、空间注意力和混合注意力三种类型。
\subsection{通道注意力机制}
输入深度网络的特征一般使用多维数据表示,通道注意力专注于挖掘不同通道间的关键,通过自适应地调整通道之间的权重,使网络模型能够更加聚焦于对后续任务有益的特征通道,从而提升模型的性能和泛化能力。压缩和激励网络(Squeeze-and-Extraction Networks, SENet)\citing{hu2018squeeze}是最具代表性的通道注意力实现模型。图\ref{SENet}为SENet的组成结构图,该方法由压缩和激励两个阶段构成。在压缩阶段,通过全局池化操作对输入多维特征进行压缩,将每个通道的信息整合成单一的数值,用于全局感受野的建模。对于维数是$H\times W \times C$的输入特征,压缩操作将其压缩为$1 \times 1 \times C$维,具体如下式所示:
\begin{equation}
    z_c=\mathbf{F}_{sq}\left( \mathbf{u}_c \right) =\frac{1}{H\times W}\sum_{i=1}^H{\sum_{j=1}^W{u\left( i,j \right)}}
\end{equation}

在激励阶段,利用全连接层和激活函数,学习得到每个通道的权重。得到的权重向量用来对原始输入特征图中的每个通道进行加权,形成加权的特征图。具体如下式所示:
\begin{equation}
    \mathbf{s}=\mathbf{F}_{ex}\left( \mathbf{z},\mathbf{W} \right) =\sigma \left( g\left( \mathbf{z},\mathbf{W} \right) \right) =\sigma \left( \mathbf{W}_2\sigma *\left( \mathbf{W}_1\mathbf{z} \right) \right)
\end{equation}
其中,$\sigma$表示ReLu激活函数\citing{nair2010rectified},$W_1 \in \mathbb{R}^{\frac{C}{r}\times C}$且$W_2 \in \mathbb{R}^{C \times \frac{C}{r}}$。

模型最终通过权重向量$s$来对输入进行重标定得到:
\begin{equation}
    \widetilde{\mathbf{x}}_c=F_{scale}\left( \mathbf{u}_c,s_c \right) =s_c\mathbf{u}_c
\end{equation}
其中,$ \widetilde{\mathbf{X}}=\left[ \widetilde{\mathbf{x}_1},\widetilde{\mathbf{x}_2},\cdots ,\widetilde{\mathbf{x}_C} \right] $, $F_{scale}\left( \mathbf{u}_c,s_c \right)$ 表示 $s_c$与特征图$\mathbf{u}_c \in \mathbb{R}^{H\times W}$的通道级乘积。


\begin{figure}[h]
    \centering
    \includegraphics[width=14cm]{pic/chapter3/SENet.jpg}
    \caption{SENet 结构图\citing{hu2018squeeze}}
    \label{SENet}
\end{figure}

\subsection{空间注意力机制}
空间注意力机制是深度学习处理图像和空间数据中的注意力机制方法。其主要目的是通过对输入数据的不同空间位置引入不同的权重,赋予模型具备灵活关注对下游任务重要的区域的能力,提升模型对空间结构的感知能力。空间注意力模块(Spatial Attention Module, SAM)\citing{woo2018cbam}是一个经典的运用空间注意力机制的方法。如图\ref{SAM}所示,SAM的主要思想是首先利用最大池化层和平均池化层获得两个全局的特征图,然后通过拼接操作将两个特征图进行拼合,再利用一个$7\times 7$的卷积核将拼合的特征图转化成单通道的特征,最后使用sigmoid激活函数\citing{}得到空间注意力权值,并与原始输入进行相乘得到最终大小与输入相同的输出。空间注意力的计算公式如下式所示:
\begin{equation}
    \begin{aligned}
        \mathbf{M}_{\mathbf{s}}\left( \mathbf{F} \right) & =\sigma \left( f^{7\times 7}\left( \left[ AvgPool\left( \mathbf{F} \right) ;MaxPool\left( \mathbf{F} \right) \right] \right) \right)
        \\
                                                         & =\sigma \left( f^{7\times 7}\left( \mathbf{F}_{\mathbf{avg}}^{\mathbf{s}};\mathbf{F}_{\mathbf{max}}^{\mathbf{s}} \right) \right)
    \end{aligned}
\end{equation}

\begin{figure}[h]
    \centering
    \includegraphics[width=14cm]{pic/chapter3/SAM.png}
    \caption{SAM 结构图\citing{woo2018cbam}}
    \label{SAM}
\end{figure}
\subsection{混合注意力机制}
混合注意力机制是综合多个注意力模块来处理数据的方法,通过对多个不同类型的注意力机制的融合,来增强深度网络模型对输入数据的建模能力,以更加灵活、全面地捕获输入数据的关键信息。卷积块注意力模块(Convolutional Block Attention Module, CBAM)\citing{woo2018cbam}是一个综合了通道和空间注意力机制的经典混合注意力方法。如图\ref{CBAM}所示,CBAM的主要网络架构有串联的通道注意力模块和空间注意力模块构成。通过依次使用通道和空间注意力模块,分别在通道和空间维度学习数据的关键信息,增强模型对输入数据的感知能力。其中,上一小节已经介绍了空间注意力模块的计算流程,而通道注意力机制的计算公式可以表示如下:
\begin{equation}
    \begin{aligned}
        \mathbf{M}_c\left( \mathbf{F} \right) & =\sigma \left( MLP\left( AvgPool\left( \mathbf{F} \right) \right) +MLP\left( MaxPool\left( \mathbf{F} \right) \right) \right)
        \\
                                              & =\sigma \left( \mathbf{W}_1\left( \mathbf{W}_0\left( \mathbf{F}_{avg}^{c} \right) \right) +\mathbf{W}_1\left( \mathbf{W}_0\left( \mathbf{F}_{max}^{c} \right) \right) \right)
    \end{aligned}
\end{equation}
其中,$\sigma$表示sigmoid函数,$W_0 \in \mathbb{R}^{C/r\times C}$、$W_1 \in \mathbb{R}^{C\times C/r}$均是多层感知机的权重参数。

因此,CBAM的计算流程可以表示为:
\begin{align}
    \mathbf{F}\prime=\mathbf{M}_{\mathbf{c}}\left( \mathbf{F} \right) \otimes \mathbf{F}
    \\
    \mathbf{F}''=\mathbf{M}_{\mathbf{S}}\left( \mathbf{F}\prime \right) \otimes \mathbf{F}\prime
\end{align}
其中,$\otimes$表示逐元素乘法。

\begin{figure}[h]
    \centering
    \includegraphics[width=14cm]{pic/chapter3/CBAM.jpg}
    \caption{CBAM 结构图\citing{woo2018cbam}}
    \label{CBAM}
\end{figure}

\section{基于双通道注意力的极化信息提取方法}
考虑到极化SAR图像原始数据可以由极化散射特征和极化目标分解特征进行表征,存在数据量大、信息复杂度高的问题,如果直接简单堆叠使用,甚至会导致模型性能下降。鉴于不同的地物目标之间散射特性存在差异,因此对极化SAR特征的重新标定校准是必不可少的,根据目标散射特性,自适应地增强有效特征,而抑制无效特征。为了高效地提取极化SAR数据中的有效信息,本章提出了一种基于双通道注意力的极化信息提取方法,旨在通过注意力机制来捕捉原始特征中的关键信息,充分考虑了不同极化特征之间的差异与联系,注重不同极化通道或空间中的关联性和权重分布,从而提高信息表征的准确性和效率。该方法可以作为一种即插即用的插件式网络结构,应用在后续的目标检测、地物分类任务中,为其提供更为可靠的基础极化信息表示。下面将详细介绍基于双通道注意力的极化信息提取方法的设计原理和实现步骤。

\subsection{双通道注意力极化信息提取网络框架}
% 图\ref{DPEN_framework}是基于双通道注意力的极化信息提取算法示意图。该极化信息提取模型由两个注意力通道构成:极化目标分解通道和极化散射特征通道。该方法的主要思路是:首先通过设计双注意力通道的结构,将目标分解特征和散射特征隔离输入,保证模型对不同层次的特征具有感知能力。然后,在每一个通道内,依次采用空间注意力和通道注意力来捕获输入特征的关键信息。为了强化极化信息的感知能力,设计了极化注意力调整模块,聚合不同层次的关键信息,对得到的注意力权重进行修正。其次,为了进一步提升模型的空间信息学习能力,引入跨空间学习模块,聚合不同尺寸的极化特征。最后,将两个通道的极化信息拼接形成最终的注意力增强的极化特征,用于下游分类任务的输入。

图\ref{DPEN_framework}是基于双通道注意力的极化信息提取算法示意图。该极化信息提取方法主要目的是充分考虑散射特征和目标分解特征的关联性,通过设计合理网络结构来对两类信息的有效提取和整合。该算法框架的主要思路是:首先,通过双通道的结构形式,分别处理极化散射特征和目标分解特征,以实现对不同信息的有针对性提取,避免相互之间的干扰。其中一个通道专注于提取散射特征中的有效信息,而另一个通道专门处理目标分解特征。其次,在每个通道内部,采用通道注意力和空间注意力模块,以捕捉关键信息并增强空间、通道关键位置的权重,从而优化极化信息提取的准确性和处理效率。然后,设计了极化注意力调整模块,旨在通过极化散射的一致性修正两个通道之间的注意力权重,增强两者之间的协同作用。同时,为了对空间特征信息的建模,每个通道内部设置两个不同尺寸的分支,通过跨空间学习模块来保证模型对极化信息的全面感知能力。最后,通过拼接操作对两个通道中重新标定的特征进行融合输出,得到最终的有效极化信息表示,为后续任务提供可靠的数据基础。

\begin{figure}[h]
    \centering
    \includegraphics[width=14cm]{pic/chapter3/DPEN_framework.png}
    \caption{基于双通道注意力的极化信息提取算法示意图}
    \label{DPEN_framework}
\end{figure}


综合上述算法框架,基于双通道注意力的极化信息提取方法的主要步骤可以表示为以下几个步骤:

(1)极化原始特征提取。极化原始特征包含极化散射特征和极化目标分解特征两种不同层次的特征。利用不同的极化分解方法从极化SAR原始数据中提取出目标分解特征目标分解特征$x_D \in \mathbb{R}^{h\times w \times c_D}$。极化目标分解特征参数具体如表\ref{decomposision feature}所示。
\begin{table}[h]
    \caption{选用的极化目标分解特征}
    \begin{tabular}{cc}
        \hline \hline
        目标分解特征 & 描述             \\ \hline
        $\left| a \right|^2,\left| b \right|^2,\left| c \right|^2$
               & Pauli分解        \\ \hline
        $H,A,\alpha$
               & $H/A/\alpha$分解 \\ \hline
        $P_{hs},P_{hd},P_{hv}$
               & Freeman分解      \\ \hline \hline
    \end{tabular}
    \label{decomposision feature}
\end{table}

散射特征$x_S \in \mathbb{R}^{h\times w \times c_S}$,由极化相干矩阵中的元素构成,具体表示为:
\begin{equation}
    \begin{aligned}
        % TODO:
        F_S=\{T_{11},T_{22},real(T_{33}),real(T_{12}),real(T_{13}),
        \\
        image(T_{23}),image(T_{33}),image(T_{12}),image(T_{13}),image(T_{23})\}
    \end{aligned}
\end{equation}
其中,$real(\cdot)$表示取实数部分,$image(\cdot)$表示取虚数部分。

(2)极化关键信息提取。在两个通道中,利用通道注意力模块和空间注意力模块,提取关键的极化信息。通道注意力模块用于提取各通道内部的显著特征,而空间注意力模块用于增强空间位置重要的数据。通过为不同的通道与空间分配不同的权重,调整不同权重值的大小来改变对应空间与通道的特征在特征向量中的比值,增强不同散射体之间的联系。对于输入的极化特征$x_i \in \mathbb{R}^{h \times w \times c}$,空间和通道注意力模块输出不同空间通道的重要程度,如下式所示:
\begin{align}
    s_i=Attention_s\left( x_i \right)
    \\
    c_i=Attention_i\left( x_i \right)
\end{align}
其中,$Attention_s$和$Attention_c$分别表示空间和通道注意力计算函数,$i \in {1,2}$表示散射特征或目标分解特征。

(3)极化特征引导的注意力修正。为了整合目标分解通道和极化散射通道全局信息,通过该极化特征引导的注意力修正模块来增强两个不同类型极化信息的关联,以获取更全局、更丰富的极化信息。该模块以目标分解注意力权重和散射特征注意力权重为输入,利用卷积与乘加的方式,来对两种信息进行整合,最后通过Sigmoid激活函数将联合的权重映射为修正后的注意力权重。

(4)跨空间特征学习。尽管通过空间、通道注意力方法能够构建起空间中的依赖关系,但是仅仅考虑了通道分支和空间分支的表示能力,对于不同空间维度方向的关系表示还有所欠缺。通过设计$3\times 3$和$1 \times 1$两个卷积分支结构,用于对全局信息编码和远程依赖关系建模。

(5)极化信息聚合。通过上述步骤之后,两个通道得到修正的极化特征,通过拼接操作将两种类型的特征进行拼接,得到最终的极化信息输出,用于后续任务的输入。

最后将上述双通道注意力极化信息提取算法伪代码总结为表\ref{}。

\begin{algorithm}[H]
    \KwData{this text}
    \KwResult{how to write algorithm with \LaTeX2e}
    极化特征表示\;
    空间、通道注意力机制捕捉关键信息\;
    极化特征引导的注意力权重修正\;
    跨空间极化特征聚合\;
    特征拼接\;
    后续分类\;
    \caption{双通道注意力极化信息提取算法}
\end{algorithm}


\subsection{空间和通道注意力模块}
引入空间和通道注意力模块,旨在进一步提升对应层次特征通道中特征的可鉴别特性。通过通道注意力自适应地校准每个通道特征图的权重、空间注意力来捕捉像素级的空间信息,在重要的特征上得到增强,而在不相关的特征上权重较小,从而改善信息提取模型的性能。本文采用(spatial and channel Squeeze Excitation, scSE)\citing{}中的空间通道注意力结构。图\ref{scSE}为scSE的结构图,包含了两个并行的模块,分别是用于提取通道注意力的cSE(channel Squeeze Excitation, cSE)和用于提取空间注意力的sSE(spatial Squeeze Excitation, sSE),输入的特征图$F$经过这两个模块的重新校准,输出校准后的特征图$F_{sSE}$和$F_{cSE}$。基于以上的校准模型,能够学习到空间、通道两种维度上更加有意义的信息。

图\ref{DPEN_WFM}展示了极化注意力调整模块详细结构。
\begin{figure}[h]
    \centering
    \includegraphics[width=10cm]{pic/chapter3/scSE.jpg}
    \caption{scSE结构图}
    \label{scSE}
\end{figure}
如图\ref{scSE}所示,相比于SE模块的全连接层,sSE空间注意力模块采用卷积操作来获取全局信息,从而赋予其更多的非线性捕捉能力,更好的拟合不同空间位置之间的复杂相关性,显著减少参数量和计算量。如图\ref{scSE}所示,sSE模块首先对每个通道素具进行压缩,利用$1\times 1 \times 1$卷积,将输入的特征图$F$的特征维度从$H \times W \times C$压缩为$H \times W \times 1$,然后利用Sigmoid激活函数得到空间注意力系数图。系数图中的每一个像素点代表了所有通道的特征图在该空间位置信息的重要程度。基于空间注意力系数图能够使模型更加关注与任务相关的空间位置信息,而抑制不相关的位置信息。

cSE模块首先对输入的特征图在特征通道上进行压缩,通过全局平均池化的方式,将特征维度为$H \times W \times C$的输入特征压缩为$1 \times 1 \times C$维度。全局平均池化操作对输入特征图的空间依赖性进行拆解,并通过逐个学习每个通道生成反映各个通道重要性的特征图。随后,这些特征图经过两个卷积层进行信息处理,最终被转换为$C$维向量。经过Sigmoid激活函数处理后,得到相应的通道注意力权重系数,该系数用于表示不同通道特征图的重要性。利用通道注意力权重系数,动态调整每个通道的重要程度,以更加精准地捕捉有助于任务的特征信息,提升模型对关键特征的敏感性。

\subsection{极化注意力修正模块}
图\ref{DPEN_WFM}展示了极化注意力调整模块详细结构。
\begin{figure}[h]
    \centering
    \includegraphics[width=14cm]{pic/chapter3/DPEN_WFM.png}
    \caption{极化注意力修正模块}
    \label{DPEN_WFM}
\end{figure}

极化信息引导的注意力修正模块利用两种不同类型的注意力权重,通过极化权重与散射权重之间的相关引导,进行修正。该模块的表达式如下所示:


\subsection{跨空间学习模块}
跨空间学习模块的网络结构如图\ref{DPEN_CSL}所示。跨空间学习模块提供了一种不同空间维度方向的极化信息聚合方法,来实现多尺度下的极化特征聚合。引入两个分支的张量,分别是$1\times 1$分支的输出和$3 \times 3$分支的输出。随后利用二维全局平均池化对$1\times 1$分支的输出中的全局极化空间信息进行编码,用于编码全局信息和建模远程依赖关系。二维全局池化操作可以表示为:
\begin{equation}
    z_c=\frac{1}{H\times W}\sum_{j}^{H}\sum_{i}^{W}x_c(i,j)
\end{equation}

在以上二维全局平均池化的输出处采用二维高斯映射的自然非线性函数Softmax来拟合线性变换,进而提升计算效率。通过将上述并行处理的输出与矩阵点积运算相乘,得出了第一个空间注意力图。同样利用二维全局平均池化在$3\times 3$分支编码全局空间信息,将每组内的输出特征映射计算为生成的两个空间注意力权重值的集合,然后使用Sigmoid函数映射成空间位置对应的权重关系。通过捕获像素级的成对关系,突出显示所有像素的全局上下文信息。
\begin{figure}[h]
    \centering
    \includegraphics[width=14cm]{pic/chapter3/DPEN_CSL.png}
    \caption{CSL 结构图}
    \label{DPEN_CSL}
\end{figure}


\section{实验结果与分析}
\subsection{实验模型介绍}
本章实验所使用的计算环境为一台CPU为Intel Core i7-8700K和配备了NVIDIA GPU(GeForce RTX 3090, 24G)的计算机设备。操作系统采用Ubuntu 20.04 LTS。深度学习框架选择PyTorch,版本为1.9.0,同时依赖CUDA深度神经网络库(cuDNN)版本8.0.5。在科学计算方面,实验使用NumPy库,版本为1.19.5。

学习率作为深度学习模型训练的关键参数之一,其值的选择对于模型的收敛速度至关重要。在训练过程中,如果学习率设置的过大或者过小,均会可能给模型的分类准确度产生负面影响。通常情况下,在模型的训练初始阶段,采用较大的学习率能够使模型快速收敛到最优点附近。随着训练的执行,逐渐减小学习率,以更加精确地接近嘴有点。在本章所采用的模型中,初始学习率为$3\times 10^{-4}$,在训练至第30至60个epoch期间,学习率经过衰减变为原值的0.1倍。这里的epoch表示训练集中素有样本完成一次正向传递和反向传播的过程,本章中模型训练的epoch设置为100。Batch size表示每次训练中选择的样本数量,其值的大小会影响网络的优化速度和执行效率。在本章模型中,batch size被设置为64。

实验中,使用两组真实极化SAR数据集分别是荷兰Flevoland区域和德国Oberpfaffenhofen地区数据来验证本章方法的有效性,并利用常规的性能指标总体分类准确率(OA)、各个类别分类准确率和Kappa系数对分类结果进行数值量化。同时对分类的可视化结果进行视觉评估。在给定的极化SAR标准数据集中,一部分像素是没有标签的,所以在计算分类准确率的时候只统计数据集中那些有标签的样本被正确分类的百分比,并认为该指标可以表征数据集中整体的分类性能。

本章实验在双通道注意力极化信息(Dual-Attention Polarization Information Extraction, DAPIE)提取方法基础上,构建端到端的极化特征提取与分类方法,记为DAPIE-CNN。DAPIE-CNN的结构如图\ref{}所示,分为极化信息提取和分类器两个部分。DAPIE模块作为模型的极化信息提取方法,用于对输入的高维极化特征进行有价值信息的激发和无价值信息的抑制。分类模块是对提取的极化信息进行精确的分类。

输入的极化特征如表\ref{pol-features}所示。
\begin{table}[ht]
    \caption{使用的极化特征列表}
    \resizebox{\linewidth}{!}{
        \begin{tabular}{ccc}
            \hline \hline
            极化特征         & 特征参数                                                & 特征数量 \\
            \hline
            Huynen       & T11,T22,T33                                         & 3    \\
            Freeman      & Freeman2(Vol,Ground),Freeman3(Odd,Dbl,Vol)          & 5    \\
            Cloude       & T11,T22,T33                                         & 3    \\
            H/A/$\alpha$ & alpha,anisotropy,beta,delta,entropy,gamma           & 6    \\
            Yamaguchi    & Yamaguchi3(Odd,Dbl,Vol),Yamaguchi4(Odd,Dbl,Vol,Hlx) & 7    \\
            Vanzyl       & Odd,Dbl,Vol,Hlx,Dbl-Hlx,wire                        & 6    \\
            \hline \hline
        \end{tabular}
    }
    \label{pol-features}
\end{table}

从输入的高维原始极化特征出发,利用DAPIE模块可以获得全局信息并且嵌入到分类器中。因此原始极化特征中有价值的信息被激发,而没有价值的信息被抑制。当具备了重新校正的极化特征之后,设计基于卷积神经网络(Convolution Neural Network, CNN)实现极化SAR图像的分类。在本节的实验验证方法中,遵循基于CNN的极化SAR分类的一般范式,使用一个类似vgg的卷积架构来拟合模型的特征输入。VggNet通过$3 \times 3$大小的堆叠卷积对经典的CNN进行改进,以获得更好的性能。分类模块的具体网络结构如表\ref{CNN-Vgg}所示。
\begin{table}[ht]
    \caption{分类模块网络结构}
    \begin{tabular}{ccc}
    \end{tabular}
    \label{CNN-Vgg}
\end{table}

以交叉熵损失函数\citing{}为目标,通过反向传播算法训练模型参数。交叉熵损失时分类问题中最常用的损失函数之一,衡量了分类模型的预测值与真实标签之间的差异性,是一种用于优化分类模型的目标函数。

综上所述,本章的DAPIE-CNN方法将输入的原始极化特征$x$映射为预测概率$p\in \mathbb{R}^{C}$,其中,$C$表示类别的个数。$x$对应的中心像素预测标签可以通过选择概率最高的类别,即向量$p$的最大值索引来预测。其计算公式如下:
\begin{equation}
    H(p,q)=\sum_{i=1}^{n}p(x_i)log(q(x_i))
\end{equation}
其中,$p(x)$表示真实值分布概率,$q(x)$表示模型预测分布概率。交叉熵值的变化与模型的训练效果密切相关,优越的训练效果会让预测概率分布逐渐趋近于真实值概率分布,相应的交叉熵值会逐渐减少。Sigmoid和Softmax损失函数是两个被广泛应用的交叉熵损失函数。Sigmoid损失函数主要应用于多标签分类任务,其中分类目标可以同时拥有多个标签。这一损失函数模拟了模型对多个独立事件的概率预测,每个事件的概率值落在$[0,1]$区间内。Softmax损失函数被广泛应用在多类别分类任务重,其中每个样本仅能关联一个类别。Softmax函数将预测模型的原始输出转化为表示类别概率的分布,确保所有类别的概率之和为1,并且模型的输出是互斥的。本章的极化SAR图像目标分类任务属于多分类语义分割任务,每个像素都有唯一的正确类别。因此选择多分类交叉熵损失函数,即Softmax损失函数作为模型的损失函数,为模型训练提供有力的优化目标。

为了对本章提出的极化信息提取方法进行全面地评估和对比,选择了多种替代方案进行比较,主要涉及两个方面的变化:一方面是对特征输入的处理改变,另一方面是对极化信息提取模块的替代策略。首先在特征输入方面,验证了不同极化特征表示对目标分类任务的影响。利用本章提出的基于双通道注意力的极化信息提取方法结合CNN分类模块进行分类、仅使用极化相干矩阵中的元素结合CNN分类方法(记为CNN-T)、仅使用极化目标分解特征结合CNN分类方法(记为CNN-P)、基于散射特征和分解特征简单叠加结合CNN的分类方法(记为CNN-F)作为不同的对比方法,验证本章方法在极化特征表示方面的有效性。其次是在极化信息提取模块层面进行替换,利用基于压缩和激励网络结合CNN分类方法(记为CNN-SE)和基于空间通道注意力结合CNN分类方法(记为CNN-CBAM)作为不同的对比方法,验证本章方法在极化特征表示方面的优越性。在每组实验中,从每个类别选择1\%的带标签像素,以这些带标签像素为中心,在其周围利用$15 \times 15$的窗口截取图像,形成训练集的样本表示。

\subsection{精度评价方法}
精度评价是对实际数据和模型分类结果进行比较的重要步骤,旨在确定分类模型的准确性,是衡量分类结果可靠性的关键指标。混淆矩阵(Confusion Matrix)通常作为遥感图像分类准确性能的评判指标,并且可以通过混淆矩阵计算得到多种常用的评价参数指标,例如总体分类准确率(Overall Accuracy, OA)、各个类别分类准确率、各个类别平均分类准确率(Average Accuracy, AA)、Kappa系数等。

混淆矩阵是一个$n \times n$的矩阵,其中$n$表示数据集的类别数量。混淆矩阵的行表示实际类别,列表示预测类别。其中,每个元素$(i, j)$表示实际属于类别$i$的样本被预测为类别$j$的数量。混淆矩阵主对角线元素表示被正确分类的样本,非主对角线表示分类错误的样本。

在极化SAR图像分类结果精度评价中,可以基于混淆矩阵定义以下指标:

1.总体分类准确率(OA):
\begin{equation}
    OA=\frac{\mbox{主对角线元素之和}}{\mbox{混淆矩阵所有元素之和}}
\end{equation}

2.生产者精度:
\begin{equation}
    \mbox{生产者精度}=\frac{\mbox{类别对应的主对角线元素}}{\mbox{类别所在列总和}}
\end{equation}

3.使用者精度。
\begin{equation}
    \mbox{使用者精度}=\frac{\mbox{类别对应的主对角线元素}}{\mbox{类别所在行总和}}
\end{equation}

4.错分误差。
错分误差是指被分类模型错误地划分为用户感兴趣的类别,实际上属于另一类别的样本数量,反映了模型在预测时产生的误报情况。
\begin{equation}
    \mbox{错分误差}=1-\mbox{使用者精度}
\end{equation}

5.漏分误差。
漏分误差是指本应该属于地表真实分类的样本,但是由于模型未能正确分类而被判为其他类别的数量,反映了模型在预测时产生的漏报情况。
\begin{equation}
    \mbox{漏分误差}=1-\mbox{生产者精度}
\end{equation}

6.Kappa系数
Kappa系数是一种通过多元统计方法来评价分类精度的指标,旨在量化分类模型的性能相对于完全随机分类的优越性。该系数通过考察混淆矩阵的对角线元素以及总体分布情况,提供了对分类结果误差的全局度量。具体计算公式如下:
\begin{equation}
    K=\frac{p_0-p_e}{1-p_e}
\end{equation}
其中,$p_0$表示总体分类精度,由主对角线元素之和除以所有样本数量计算得到;$p_e$表示某一个类别地表真实样本总数与该类中被分类样本总数之积对所有类别求和除以总样本数的平方。将混淆矩阵中的具体元素带入上式,可以得到:
\begin{equation}
    K=\frac{N\sum_{i=1}^{r}{{x_i}_i}-\sum_{i=1}^{r}{\left( {x_i}_+x_{+i} \right)}}{N^2-\sum_{i=1}^{r}{\left( {x_i}_+x_{+i} \right)}}
\end{equation}

Kappa系数的大小可以用来表示分类的精度性能,表\ref{kappa}描述了Kappa系数与模型的分类精度的映射关系。
\begin{table}[ht]
    \caption{Kappa统计值与分类精度映射关系}
    \begin{tabular}{cc}
        \toprule[1.5bp]
        Kappa系数 & 分类精度 \\
        \midrule[0.75bp]
        <0      & 较差   \\
        0-0.2   & 差    \\
        0.2-0.4 & 正常   \\
        0.4-0.6 & 好    \\
        0.6-0.8 & 较好   \\
        0.8-1   & 非常好  \\
        \bottomrule[1.5bp]
    \end{tabular}
    \label{kappa}
\end{table}

\subsection{AIRSAR Flevoland数据实验}
实验数据集选择NASA/JPL于1989年在Flevoland区域采集得到的全极化数据。该数据集是荷兰的一个农业区域遥感数据,作为基准数据集广泛应用于极化SAR土地覆盖目标分类研究中。该图像大小为$1024 \times 750$ 像素,共有15种农作物类别,包括茎豆、豌豆、森林、苜蓿、小麦、甜菜、土豆、裸土、草、油菜籽、大麦、水和少量建筑物。各个农作物目标类别之间的差异较小,相似性较强,因此分类难度较大,容易出现错分漏分的现象。图\ref{flevoland_pauli}和图\ref{flevoland_gt}分别展示了AIRSAR Flevoland数据集的Pauli分解伪彩图像以及对应的地面真值标签图像。表\ref{flevoland_smaple}展示了该数据集中每个类别带标签的样本数量。
\begin{figure}[ht]
    \subfloat[]{
        \label{flevoland_pauli}
        \includegraphics[width=7.04cm]{pic/chapter3/fle/pauli.png}
    }
    \subfloat[]{
        \label{flevoland_gt}
        \includegraphics[width=7.04cm]{pic/chapter3/fle/gt.png}
    }

    \subfloat[]{
        \label{pice}
        \includegraphics[width=9.04cm]{pic/chapter3/fle/label.png}
    }
    \caption{Flevoland地区实验数据集。(a)Pauli分解伪彩色图像;(b)实验数据地面真值;(c)颜色与类别对应关系}
    \label{fig2}
\end{figure}

\begin{table}[h]
    \caption{Felvoand地区实验数据集有标签样本数量}
    \begin{tabular}{|c|c|c|c|c|c|c|c|c|}
        \hline 类别 & 干豆   & 大麦    & 裸土    & 土豆    & 甜菜    & 小麦 1  & 豌豆    & 苜蓿
        \\
        \hline 数目 & 6338 & 7595  & 5109  & 16156 & 10033 & 11159 & 9582  & 10181 \\
        \hline 类别 & 草地   & 小麦 2  & 油菜籽   & 小麦 3  & 建筑物   & 森林    & 水域    &       \\
        \hline 数目 & 7058 & 16386 & 13863 & 22241 & 735   & 18044 & 13232 &       \\
        \hline
    \end{tabular}
    \label{flevoland_smaple}
\end{table}

图\ref{fig:fle_res}展示了各个对比方法的可视化分类结果。根据图\ref{fig:fle_T},仅使用散射特征的分类方法,在大麦、草地和小麦区域都有较多的错分样本,这是因为只使用了散射特征而忽略了目标分解特征,没有全面综合利用所有的极化信息导致的。图\ref{fig:fle_P}展示了仅使用目标分解特征的分类结果,在甜菜、小麦区域也存在大量的错分样本,这可能是由于没有综合使用极化信息导致的,相干矩阵代表的散射特征是极化SAR中最基本、重要的特征。图\ref{fig:fle_F}展示的简单叠加散射特征与目标分解特征的分类结果,该方法的分类结果要优于前面两种方法,类间错分孤立点相对减少,具有更少的错分样本,这反映了综合利用极化信息对目标分类具有一定的优势。图\ref{fig:fle_SE}与图\ref{fig:fle:CBAM}展示了使用经典注意力方法的分类结果图,相比于直接叠加特征,并没有带来明显的分类性能提升,少量的错分孤立点依然存在。图\ref{fig:fle_DP}展示了基于双通道注意力方法的分类结果,可以看出本方法分类结果更加平滑,错分像素减少,特别是是在小麦、草地区域,这也验证了本章方法结合散射特征和目标分解特征的有效性,证明本章方法提取的极化信息表示是优越的。
\begin{figure}[ht]
    \subfloat[]{
        \label{fig:fle_T}
        \includegraphics[width=4.74cm]{pic/chapter3/fle/CNN-T.png}
    }
    \subfloat[]{
        \label{fig:fle_P}
        \includegraphics[width=4.74cm]{pic/chapter3/fle/CNN-P.png}
    }
    \subfloat[]{
        \label{fig:fle_F}
        \includegraphics[width=4.74cm]{pic/chapter3/fle/CNN-F.png}
    }

    \subfloat[]{
        \label{fig:fle_SE}
        \includegraphics[width=4.74cm]{pic/chapter3/fle/CNN-T.png}
    }
    \subfloat[]{
        \label{fig:fle_CBAM}
        \includegraphics[width=4.74cm]{pic/chapter3/fle/CNN-P.png}
    }
    \subfloat[]{
        \label{fig:fle_DP}
        \includegraphics[width=4.74cm]{pic/chapter3/fle/CNN-F.png}
    }
    \caption{AIRSAR Flevoland地区数据分类可视化结果图。(a) CNN-T; (b) CNN-P; (c) CNN-F; (d) CNN-SE; (e) CNN-CBAM; (f) 本章方法}
    \label{fig:fle_res}
\end{figure}

为了进一步探索极化信息提取模块的性能,引入t-SNE(t-distributed Stochastic Neighbor Embedding)\citing{}特征分布图作为衡量极化信息提取模块的可视化评估指标。t-SNE是一种非线性降维技术,能够有效地将高维数据映射到二维或三维空间,以便更直观地观察样本在特征空间的分布。

图\ref{fig:fle-tSNE}展示了不同方法特征特征使用t-SNE可视化的分布情况。根据图\ref{fig:fle_t_t}、图\ref{fig:fle_t_p}和图\ref{fig:fle_t_f}所展示的仅散射特征、仅分解特征和直接堆叠特征三种不同的特征分布情况,可以看出大多数类别相互之间有交叠的情况,并且类内的样本分布较为散乱,因而容易造成类间错分的情况,导致分类性能优先。而图\ref{fig:fle_t_se}和图\ref{fig:fle_t_cbam}展示的使用经典的注意力信息提取方法的特征分布图,可以看出经典的注意力方法由于没有考虑极化SAR的数据特性,不同类别之间的重叠情况依然存在,并且类内样本特征分布散乱,证明了在信息提取时不考虑极化SAR数据特征对分类模型的性能提升是有限的。根据图\ref{fig:fle_t_dp}所展示的基于双通道注意力的极化信息提取方法的特征分布图,可以看到相比于其他的特征分布情况,类间的交叠现象有了明显的改善,同时类内样本特征分布更加紧凑,提升类间可分性,进而带来分类性能上的提升。


\begin{figure}[ht]
    \subfloat[]{
        \label{fig:fle_t_t}
        \includegraphics[width=4.74cm]{pic/chapter3/fle/tSNE-T.png}
    }
    \subfloat[]{
        \label{fig:fle_t_p}
        \includegraphics[width=4.74cm]{pic/chapter3/fle/tSNE-P.png}
    }
    \subfloat[]{
        \label{fig:fle_t_f}
        \includegraphics[width=4.74cm]{pic/chapter3/fle/tSNE-F.png}
    }

    \subfloat[]{
        \label{fig:fle_t_se}
        \includegraphics[width=4.74cm]{pic/chapter3/fle/tSNE-SE.png}
    }
    \subfloat[]{
        \label{fig:fle_t_cbam}
        \includegraphics[width=4.74cm]{pic/chapter3/fle/tSNE-CBAM.png}
    }
    \subfloat[]{
        \label{fig:fle_t_dp}
        \includegraphics[width=4.74cm]{pic/chapter3/fle/tSNE-DP.png}
    }

    \subfloat[]{
        \includegraphics[width=10.04cm]{pic/chapter3/fle/tSNE-label.png}
    }
    \caption{AIRSAR Flevoland地区数据不同方法特征分布。((a) CNN-T; (b) CNN-P; (c) CNN-F; (d) CNN-SE; (e) CNN-CBAM; (f) 本章方法; (g) 颜色与类别对应关系}
    \label{fig:fle-tSNE}
\end{figure}

表\ref{tab:fle_result}展示了不同方法的分类数值结果。仅使用散射特征的分类方法,在草地和建筑物区域的分类准确率较低,分别为89.41\%和91.17\%,这可能是因为仅使用相干矩阵作为极化特征的分类方法,并不能完全反应这两类地物目标的散射特性,导致分类准确率较低。仅使用目标分解特征作为特征的分类方法,在小麦区域取得了较好的分类情况,这是因为目标分解特征中包含有具体物理意义的极化表征,但是在建筑物油菜籽区域的分类准确率较低,反映了目标分解特征与散射特征相互补充,都是必不可少的极化特征表示方法。直接堆叠极化特征的分类方法在分类总体准确率、平均准确率以及大麦等区域准确率反而下降,这表明了直接堆叠极化特征导致特征维数增加,分类器无法有效分清两种类型的主次关系,导致分类准确率有所下降。根据基于经典注意力方法的分类结果,尽管分类结果准确率指标有小幅度提升,但是实质上没有明显的数值提升,也是因为经典的注意力方法并没有考虑极化特征的特点,并不适用于极化信息提取领域。根据本章方法的分类数值结果,在大多数的类别都具有最高的分类准确率,并且在总体准确率、平均准确率、Kappa系数分别提升1.02\%、0.8\%和1.14\%,表明了基于双通道注意力的极化信息提取方法对于极化SAR目标分类具有优势,验证了本章方法的有效性和优越性。

\begin{table}[ht!]
    \label{tab:fle_res}
    \caption{AIRSAR Flevoland地区数据分类数值结果(\%)}
    \renewcommand\arraystretch{1.0}
    \begin{tabular}{cccccccc}
        \toprule[1.5bp]
        序号                        & 类别    & CNN-T          & CNN-P          & CNN-F          & CNN-SE         & CNN-CBAM       & 本章方法           \\
        \midrule[0.75bp]
        1                         & 建筑物   & 91.17          & 90.2           & 94.01          & \textbf{99.79} & 94.14          & 97.67          \\
        2                         & 油菜籽   & 94.33          & 91.78          & 89.78          & 94.54          & 92.32          & \textbf{95.98} \\
        3                         & 甜菜    & 96.57          & 96.62          & 96.99          & 97.92          & 95.74          & \textbf{98.18} \\
        4                         & 干豆    & \textbf{99.66} & 99.26          & 98.86          & 97.08          & 98.83          & 99.15          \\
        5                         & 豌豆    & 96.72          & 97.03          & 96.54          & 99.8           & 95.72          & \textbf{98.96} \\
        6                         & 森林    & 98.13          & 97.37          & 96.42          & 98.85          & 98.61          & \textbf{98.71} \\
        7                         & 苜蓿    & 96.03          & 95.96          & 95.07          & 96.11          & 95.48          & \textbf{96.12} \\
        8                         & 土豆    & 96.61          & 97.03          & \textbf{97.54} & 97.33          & 95.59          & 97.22          \\
        9                         & 裸土    & 95.02          & 97.19          & 98.44          & 98.51          & 96.8           & \textbf{98.65} \\
        10                        & 草地    & 89.41          & 93.11          & 92.32          & 96.19          & 94.25          & \textbf{96.7}  \\
        11                        & 大麦    & 93.63          & 95.45          & 87.34          & 95.64          & 89.34          & \textbf{98.11} \\
        12                        & 水域    & 99.94          & 99.7           & 99.66          & 93.3           & 99.83          & \textbf{99.96} \\
        13                        & 小麦 1  & 97.53          & 96.74          & 97.36          & 97.27          & 97.6           & \textbf{97.65} \\
        14                        & 小麦 2  & 93.62          & \textbf{94.48} & 94.21          & 97.53          & 92.25          & 94.46          \\
        15                        & 小麦 3  & 96.88          & 97.05          & 98.06          & 94.85          & 98.51          & \textbf{99.14} \\
        \midrule[0.75bp]
        \multicolumn{2}{c}{OA}    & 96.4  & 96.41          & 95.9           & 96.76          & 96.13          & \textbf{97.82}                  \\
        \multicolumn{2}{c}{AA}    & 95.68 & 95.93          & 95.51          & 96.98          & 95.67          & \textbf{97.78}                  \\
        \multicolumn{2}{c}{Kappa} & 96.07 & 96.08          & 95.53          & 96.48          & 95.77          & \textbf{97.62}                  \\
        \bottomrule[1.5bp]
    \end{tabular}
\end{table}

图\ref{fig:fle_conf_matrix}展示了本章方法分类结果的混淆矩阵可视化图。从图中可以看到,混淆矩阵的可视化结果显示本章方法在对角线上具有较高的主对角元素,说明在大多数类别上取得了良好的分类性能。

\begin{figure}[h]
    \centering
    \includegraphics[width=10.4cm]{pic/chapter3/fle/conf-matrix.png}
    \caption{本章方法在Flevoland图像中分类结果混淆矩阵图}
    \label{fig:fle_conf_matrix}
\end{figure}


\subsection{E-SAR Oberpfaffenhofen数据实验}
该数据集是通过E-SAR机载平台在德国Oberpfaffenhofen地区拍摄获取到的L波段极化SAR数据。该数据集经过多视处理,具有高质量的全极化信息,是极化SAR目标分类研究的经典数据集之一。数据集在方位向和距离向具有$3m \times 3m$的分辨率,大小为$1300 \times 1200$,总计包含三种类型的地物目标,分别是:建筑、林地和开放区域。图\ref{fig:ober_pauli}展示了该数据集的Pauli伪彩色图像,图\ref{fig:ober_gt}展示了该数据集的人工标记的地面真值参考图。地面真值图中黑色的区域表示未标记的区域。表\ref{fig:ober_label}展示了各类地物目标样本数量情况。

\begin{figure}[ht]
    \subfloat[]{
        \label{fig:ober_pauli}
        \includegraphics[width=4.04cm]{pic/chapter3/ober/Pauli.png}
    }
    \subfloat[]{
        \label{fig:ober_gt}
        \includegraphics[width=4.04cm]{pic/chapter3/ober/gt.png}
    }

    \subfloat[]{
        \label{fig:ober_label}
        \includegraphics[width=7.04cm]{pic/chapter3/ober/label.png}
    }
    \caption{E-SAR Oberpfaffenhofen区域数据。(a)Pauli分解伪彩色图像;(b)实验数据地面真值;(c)颜色与类别对应关系}
\end{figure}

\begin{table}[ht]
    \caption{ESAR Oberpfaffenhofen数据样本数量}
    \label{tab:ober_sample}
    \centering
    \begin{tabular}{|c|c|c|c|}
        \hline
        类别 & 建筑      & 林地      & 开放区     \\ \hline
        数目 & 269,184 & 388,503 & 779,962 \\ \hline
    \end{tabular}
\end{table}

图\ref{fig:ober_res}展示了不同方法的分类结果可视化图像。根据图\ref{fig:ober_T}展示的仅使用极化相干矩阵为特征的分类结果图,尽管在开放区域的分类效果较好,但是在建筑物区域存在大量被错分的样本,将建筑物区域样本错误分类为开放区域,并且错分的现象较为严重。这可能是因为仅使用极化相干矩阵作为特征时并没有完全利用其他的目标分解特征导致的,说明了综合考虑极化SAR图像中的其他极化特性和辅助特征的在分类任务中的重要性。相比之下,图\ref{fig:ober_P}与图\ref{fig:ober_F}在建筑物区域中的错分样本相对较少,但是在各类区域中也存在着部分错分的样本。其中将开放区域中的少数像素分类为建筑物的错分情况依然存在,这反映了极化目标分解特征提供的反映目标物理散射特性的信息为分类任务带来了一定的优势,是极化SAR图像中的重要的基本特征。图\ref{fig:ober_SE}和图\ref{fig:ober_CBAM}展示了基于经典注意力方法的特征表示分类结果,从结果图中可以看出,分类效果仅有微小的变化,错分像素相对减少但是依然存在,这可能是极化SAR图像与光学图像存在较大的差异,经典的注意力方法并没有考虑极化SAR图像的数据特征特点。图\ref{fig:ober_DP}展示了基于本章提出的双通道注意力极化信息提取方法的分类结果图,从图中直观可以看出,在建筑物区域的错分样本明显减少,并且每个类中的小块错误区域数量也相对减少,整体分类结果得到明显改善,证明了本章方法的有效性。

\begin{figure}[ht]
    \subfloat[]{
        \label{fig:ober_T}
        \includegraphics[width=4.04cm]{pic/chapter3/ober/CNN-T.png}
    }
    \subfloat[]{
        \label{fig:ober_P}
        \includegraphics[width=4.04cm]{pic/chapter3/ober/CNN-P.png}
    }
    \subfloat[]{
        \label{fig:ober_F}
        \includegraphics[width=4.04cm]{pic/chapter3/ober/CNN-F.png}
    }

    \subfloat[]{
        \label{fig:ober_SE}
        \includegraphics[width=4.04cm]{pic/chapter3/ober/CNN-SE.png}
    }
    \subfloat[]{
        \label{fig:ober_CBAM}
        \includegraphics[width=4.04cm]{pic/chapter3/ober/CNN-CBAM.png}
    }
    \subfloat[]{
        \label{fig:ober_DP}
        \includegraphics[width=4.04cm]{pic/chapter3/ober/CNN-DP.png}
    }

    \caption{E-SAR Oberpfaffenhofen地区数据分类可视化结果图。(a) CNN-T; (b) CNN-P; (c) CNN-F; (d) CNN-SE; (e) CNN-CBAM; (f) 本章方法}
    \label{fig:ober_res}
\end{figure}

图\ref{fig:ober-tSNE}展示了不同方法数据特征使用tSNE可视化的分布情况。根据图\ref{fig:ober_t_t}、图\ref{fig:ober_t_p}、图\ref{fig:ober_t_f}展示的仅散射特征、仅分解特征和直接堆叠的三种特征分布情况,可以看出三种特征分布情况并没有本质性的差异,尽管都可以看出各类之间存在不同的分布情况,但是也能清晰看出,类间存在分布交叠的部分,这也导致了部分样本的错分情况。根据图\ref{fig:ober_t_se}和图\ref{fig:ober_t_cbam}展示的使用经典注意力信息提取方法的特征分布图,可以看出虽然类间交叠的情况有细微改善,但是依然有大量的特征存在交叠,这也映照了分类结果改善不大的结果。图\ref{fig:ober_t_dp}展示了使用本章方法的特征分布情况,可以看出类间交叠的情况得到优化,各类之间分布间距增大,增强了类间可分性,进而带来分类性能的提升。

\begin{figure}[ht!]
    \subfloat[]{
        \label{fig:ober_t_t}
        \includegraphics[width=4.04cm]{pic/chapter3/ober/tSNE-P.png}
    }
    \subfloat[]{
        \label{fig:ober_t_p}
        \includegraphics[width=4.04cm]{pic/chapter3/ober/tSNE-P.png}
    }
    \subfloat[]{
        \label{fig:ober_t_f}
        \includegraphics[width=4.04cm]{pic/chapter3/ober/tSNE-F.png}
    }

    \subfloat[]{
        \label{fig:ober_t_se}
        \includegraphics[width=4.04cm]{pic/chapter3/ober/tSNE-SE.png}
    }
    \subfloat[]{
        \label{fig:ober_t_cbam}
        \includegraphics[width=4.04cm]{pic/chapter3/ober/tSNE-CBAM.png}
    }
    \subfloat[]{
        \label{fig:ober_t_dp}
        \includegraphics[width=4.04cm]{pic/chapter3/ober/tSNE-P.png}
    }

    \subfloat[]{
        \includegraphics[width=7.04cm]{pic/chapter3/ober/tSNE-label.png}
    }
    \caption{E-SAR Oberpfaffenhofen地区数据地区数据不同方法特征分布。(a) CNN-T; (b) CNN-P; (c) CNN-F; (d) CNN-SE; (e) CNN-CBAM; (f) 本章方法}
    \label{fig:ober-tSNE}
\end{figure}

表\ref{tab:ober_res}展示了各个方法的分类数值结果。本章提出的基于双通道注意力的极化信息提取方法总体准确率达到94.6\%,相比于其他验证方法提升了3.73\%。本章方法的散射特征时,在建筑物区域的分类准确率仅为63.16\%,这可能是因为建筑物区域目标散射回波差异性大,导致仅使用散射特征时,不能完全表征建筑物目标的特性,进而导致在建筑物区域可分性较低。与仅使用目标分解特征相比,直接堆叠极化特征在平均准确率指标反而下降1.32\%,这可能是由于该方法仅仅是对两类的特征进行简单的堆叠,堆叠后的特征维数过高,分类模型对于高维特征输入无法有效利用两种不同类型特征间的主次关系。本章方法在总体准确率、平均准确率、Kappa系数等指标上均有提升,验证了本章方法能有效的提升极化特征的表征方式,提升分类准确度。

图\ref{fig:ober_conf_matrix}展示了本章方法分类结果的混淆矩阵可视化图。混淆矩阵的可视化结果显示本章方法在对角线上具有较高的主对角元素,说明在大多数类别上取得了良好的分类性能。

%TODO:
\begin{table}[ht!]
    \caption{E-SAR Oberpfaffenhofen地区数据分类数值结果(\%)}
    \label{tab:ober_res}
    \begin{tabular}{cccccccc}
        \toprule[1.5bp]
        序号                        & 类别    & CNN-T          & CNN-P & CNN-F & CNN-SE & CNN-CBAM       & 本章方法           \\
        \midrule[0.75bp]
        1                         & 建筑    & \textbf{96.68} & 83.9  & 80.1  & 80.42  & 83.92          & 90.96          \\
        2                         & 林地    & 83.35          & 91.99 & 90.5  & 92.44  & 91.61          & \textbf{92.45} \\
        3                         & 开放区域  & 89.83          & 90.04 & 91.37 & 91.63  & 91.41          & \textbf{96.89} \\
        \midrule[0.75bp]
        \multicolumn{2}{c}{OA}    & 89.56 & 88.97          & 88.52 & 89.13 & 90.27  & \textbf{94.6}                   \\
        \multicolumn{2}{c}{AA}    & 89.95 & 88.64          & 87.32 & 88.16 & 89.98  & \textbf{93.43}                  \\
        \multicolumn{2}{c}{Kappa} & 81.63 & 83.51          & 82.73 & 83.58 & 85.24  & \textbf{90.78}                  \\
        \bottomrule[1.5bp]
    \end{tabular}
\end{table}

\begin{figure}[ht!]
    \centering
    \includegraphics[width=10.4cm]{pic/chapter3/ober/conf-matrix.png}
    \caption{本章方法在E-SAR Oberpfaffenhofen图像中分类结果混淆矩阵图}
    \label{fig:ober_conf_matrix}
\end{figure}



\section{本章小结}
本章针对极化ASR图像信息利用中所面临的信息冗余问题,提出了一种基于双通道注意力的极化信息提取方法。通过构建极化散射特征通道和目标分解特征通道的双通道结构,结合空间、通道注意力机制,能够有效地提取出两种不同类型的极化特征中的关键极化信息。为了增强两种信息的关联,设计了散射特征导向的注意力修正方法,对注意力权重进行修正,从而全面聚合关键极化信息。最后设计多尺度特征学习模块,赋能模型对不同尺度极化信息的捕捉能力。最后通过视觉和数据量化两个层面进行性能评估,结果表明本章提出的方法能够有效地提升极化数据中的信息表征方式,从而提升极化SAR图像的分类精度。
\chapter{极化SAR标签噪声目标分类方法}
\section{引言}
% 问题引入
深度卷积神经网络(DCNN)由于其对大规模数据进行端到端的学习能力以及强大的特征学习能力,在极化SAR图像分类领域取得了显著的成果\citing{}。DCNN的优越分类性能依赖于数量充足的高质量标记样本,而极化SAR样本标注工作都是通过人工标记完成,很可能因为人工标注错误、自动标注辅助技术误差、地面覆盖类型不准确等因素存在错误标记的情况,引入标签噪声\citing{}。DCNN方法在存在标签噪声的训练集中训练时,由于其强大的拟合能力甚至会学习到噪声的特征,出现过拟合的现象,导致分类性能有限\citing{}。而现有的大多数基于DCNN的极化SAR图像分类方法都是在理想的训练集中完成模型训练,很少有考虑标签噪声问题。因此研究对标签噪声的极化SAR图像分类方法具有重要的价值与意义。

% 本章方法
本章着力于含标签噪声的极化SAR图像分类方法研究,提出了一种基于混合模型估计与边界增强的极化SAR图像分类方法。首先,根据噪声样本与干净样本的损失函数分布差异特性,使用混合模型对分布差异进行拟合,以估计样本的噪声概率。其次,根据极化SAR图像标签噪声更多存在于类别边界区域特性,通过捕获类别边界区域,对边界样本施加额外的惩罚,充分利用边界样本信息。最后,基于改进的自学习优化损失函数,训练过程中通过引入模型的预测来增加感知项,对损失函数进行纠正,完成模型的鲁棒参数优化过程。

\section{有限混合模型}
DCNN模型在含标签噪声样本集上训练时,总是先学习简单的分类规则,之后再花足够的时间来对错误信息进行过拟合\citing{}。相比于干净的标记样本,随机标记的错误样本需要更长的时间来学习,这意味着有噪声样本在训练的前期阶段具有更高的损失,这使得干净样本与噪声样本可以通过损失值的分布情况进行区分。由于对观测值的建模能力,有限混合模型适用于估计干净样本与噪声样本的损失分布估计的场景。

有限混合模型(Finite Mixture Model, FMM)是一种用于描述观测数据生成过程的概率模型。该模型假设观测的数据是由有限个概率分布组合而成的混合分布,而每个成分分布对应一个潜在的类别,混合模型需要考虑这些类别之间的不同权重。

在有限混合模型中,假设观测数据来自于K个不同的成分,每个成分对应一个概率分布。对于一维数据而言,有限混合模型的概率密度函数可以表示为:
\begin{equation}
    f(x)=\sum_{k=1}^{K}\pi_k \cdot f_k(x)
\end{equation}

其中,$f_k(x)$表示第$k$个成分分布的概率密度函数,$\pi_k$表示对应成分的权重,并且满足$\sum_{k=1}^{K}=1$。表示该数据的分布是由多个成分分布的加权和。

% 在参数估计过程中,通常使用期望最大化算法(Expectation-Maximization, EM)来估计混合模型的参数,包括成分分布的参数和权重。有限混合模型在聚类、密度估计等领域都有广泛的应用,尤其在处理多个子群体或模态的数据集时表现出色。
\subsection{高斯混合模型}
高斯混合模型(Gaussian Mixture Model, GMM)是一种将一个复杂的概率分布建模为多个高斯分布的线性组合的概率模型。相比于单一的高斯概率密度函数,GMM能够更加灵活地适应多峰的数据。GMM的核心思想是将观察到的数据视为多个高斯分布组成的混合体,每个高斯分布为一个组成分量,而所有的分量的加权和形成了整个的混合模型。

对于一个服从高斯分布的连续随机变量$x$,其概率密度函数(PDF)可以表示为:
\begin{equation}
    \begin{gathered}
        p\left( x \right) =\frac{1}{\left( 2\pi \right) ^{1/2}\sigma}\exp \left[ -\frac{1}{2}\left( \frac{x-\mu}{\sigma} \right) ^2 \right] =\mathcal{N} \left( x;\mu ,\sigma ^2 \right)
        \\
        \left( -\infty <x<\infty ;\sigma >0 \right)
    \end{gathered}
\end{equation}
记作$x \sim \mathcal{N}(\mu,\sigma^2)$,其中,$\mu,\sigma^2$分别表示$x$的均值与方差。

对于由多个高斯变量随机混合形成的部分,记为高斯混合分布。服从高斯混合分布的连续随机变量$x$的概率密度函数可以表示为:
\begin{equation}
    \begin{gathered}
        \begin{aligned}
            p(x) & =\sum_{m=1}^M \frac{c_m}{(2 \pi)^{1 / 2} \sigma_m} \exp \left[-\frac{1}{2}\left(\frac{x-\mu_m}{\sigma_m}\right)^2\right] \\ & =\sum_{m=1}^M c_m \mathcal{N}\left(x ; \mu_m, \sigma_m^2\right)
        \end{aligned}
        \\
        \left(-\infty<x<\infty ; \sigma_m>0 ; c_m>0\right)
    \end{gathered}
\end{equation}
其中,$c_m$表示不同混合分类的权重并且满足$\sum_{m=1}^M c_m=1$。

在上述的高斯混合分布中,涉及到一组参数,用$\Theta = \{c_m, \mu_m, \Sigma_m\}$ 表示。其中,$c_m$表示每个分布的权重,$\mu_m$表示均值,$\Sigma_m$表示协方差矩阵。在考虑包含多个混合分类的情况下,这一组参数可以用来描述整个混合分布的特征。要获得对混合分布参数的准确描述,需要根据一组假设来确定这些参数的值,通过最佳的参数,来匹配观测到的数据,进而更加准确的反映真实数据的特征。图\ref{GMM}展示了二分量的GMM模型概率密度函数示意图:

\begin{figure}[ht!]
    \centering
    \includegraphics[width=10cm]{pic/chapter4/GMM.png}
    \caption{二分量的高斯混合模型概率密度函数示意图}
    \label{GMM}
\end{figure}

\subsection{贝塔混合模型}
贝塔混合模型(Beta Mixture Model, BMM)是一种概率模型,将复杂的概率分布建模为多个贝塔分布的线性组合。贝塔分布是一个定义在区间 [0, 1] 上的概率分布,通常用于描述在二项分布的贝叶斯估计中。

对于一个服从贝塔分布的连续随机变量 $x$,其概率密度函数(PDF)可以表示为:
\begin{equation}
    \label{eq:beta1}
    p(x; \alpha, \beta) = \frac{x^{\alpha - 1} (1 - x)^{\beta - 1}}{B(\alpha, \beta)}
\end{equation}
其中,$B(\alpha, \beta) = \frac{\Gamma(\alpha) \Gamma(\beta)}{\Gamma(\alpha + \beta)}$;$\alpha$ 和 $\beta$ 是贝塔分布的参数,$\Gamma$ 表示伽玛函数。

对于由多个贝塔分布随机混合形成的部分,记为贝塔混合分布。服从贝塔混合分布的连续随机变量 $x$ 的概率密度函数可以表示为:
\begin{equation}
    \label{eq:beta}
    \begin{gathered}
        \begin{aligned}
            p(x) & = \sum_{m=1}^M c_m \frac{x^{\alpha_m - 1} (1 - x)^{\beta_m - 1}}{B(\alpha_m, \beta_m)} \\
                 & = \sum_{m=1}^M c_m \text{Beta}(x; \alpha_m, \beta_m)
        \end{aligned}
        \\
        (0 < x < 1; c_m > 0; \sum_{m=1}^M c_m = 1; \alpha_m, \beta_m > 0)
    \end{gathered}
\end{equation}
其中,$c_m$ 表示不同混合分类的权重,满足 $\sum_{m=1}^M c_m=1$。

在贝塔混合分布中,参数集为 $\Theta = \{c_m, \alpha_m, \beta_m\}$。其中,$c_m$ 表示每个贝塔分布的权重,$\alpha_m$ 和 $\beta_m$ 分别表示每个贝塔分布的形状参数。为了描述整个混合分布的特征,需要通过一组假设,从观测到的数据中确定这些参数的值。这一过程旨在使混合分布的参数最佳匹配观测到的数据,以更准确地反映实际数据的特征。图\ref{BMM}展示了二分量的BMM模型概率密度函数示意图:

\begin{figure}[ht!]
    \centering
    \includegraphics[width=10cm]{pic/chapter4/BMM.png}
    \caption{二分量的贝塔混合模型概率密度函数示意图}
    \label{BMM}
\end{figure}

\section{标签噪声下极化SAR目标分类方法}
\subsection{标签噪声问题建模}
% 由于极化SAR的复杂成像机制,获取足够数量的训练样本是一项非常耗时和耗力的任务。而且,现有数据集中的标注虽然可能近似完美,但实际上存在一些错误标记的样本。这种标签噪声可能由于人工标记的错误、边界区域的模糊性以及场景图像的错误等原因产生。在使用混合标签噪声的训练集进行训练时,模型可能学习到噪声样本的特征,从而对原有的分类规则造成干扰,导致过拟合的问题。
给定一个包含N个类别的干净PolSAR数据的训练集,表示为:
\begin{equation}
    (X, Y^{(c)})=\{(x_1,y_1^{(c)}),\ldots,(x_N,y_N^{(c)})\}
\end{equation}
该数据来源于分布$D=X \times Y$。假设存在一个函数$\mathcal{F}:Y^{(c)} \rightarrow Y^{(n)}$,该函数向标签$Y^{(c)}$引入噪声,将$\mathcal{F}$应用于$(X, Y^{(c)})$得到一个带有噪声的训练数据集,表示为:
\begin{equation}
    (X, Y^{(n)})=\{(x_1,y_1^{(n)}),\ldots,(x_N,y_N^{(n)})\}
\end{equation}
其中$Y^{(n)}$是原始干净标签$Y^{(c)}$和被污染的标签$Y^{(n)} \neq Y^{(c)}$的组合。

假定标签$y_i^{(c)}$被噪声函数污染成$y_i^{(n)}$,存在对称噪声源与非对称噪声源两种类型。对称噪声源模型是将真实标签以相同的概率随机翻转到其他类别,服从均匀分布,表达式如下:
\begin{equation}
    P(y_i=y_i^{(n)} \mid y_i^{(c)})=\frac{1}{N-1}
\end{equation}

非对称噪声源模型是按照根据某种固定的规则进行标签映射操作,将真实标签以随机概率翻转到其他的某个类别。图\ref{fig:noisy_type}展示了两种类型噪声源的噪声转移矩阵示意图。
\begin{figure}[ht!]
    \subfloat[]{
        \includegraphics[width=7.04cm]{pic/chapter4/Random.png}
    }
    \subfloat[]{
        \includegraphics[width=7.04cm]{pic/chapter4/One.png}
    }
    \caption{30\%比例噪声转移矩阵示意图。(a)对称噪声 (b)非对称噪声}
    \label{fig:noisy_type}
\end{figure}

从干净分布$D$中随机选择一个测试集$(X_T, Y_T)$。目标是使用带有噪声的数据集$(X, Y^{(n)})$训练一个分类器模型$M:X \rightarrow Y$,以确保对干净测试集$(X_T, Y_T)$的鲁棒性泛化,并且在训练过程中无法访问原始的干净标签$Y_c$。

% 深度网络模型在混合噪声的数据集上训练时,总是会先学习简单的干净样本特征,之后再花更长的时间来拟合随机的标签噪声。这意味着噪声样本在训练的前期具有更高的损失,这样可以通过损失的分布来区分噪声和干净样本。

% 本章中,为了解决极化SAR图像中的标签噪声问题,提出了一个端到端的标签噪声条件下的极化SAR图像分类方法,基于贝塔混合分布引导的标签噪声下极化SAR图像分类方法。首先利用贝塔混合概率模型拟合噪声样本、干净样本的损失函数分布以区分噪声样本和干净样本。然后,设计动态损失函数,结合模型的预测类别对噪声样本进行校正。同时,通过对边界样本损失增强实现模型对边界样本的关注提升。将上一章的插件式极化信息提取方法作为本分类任务的前置模块,旨在应用深层次极化信息来辅助该特定的分类任务,并提升该任务下的分类精度。

% 近年来,各种应对于标签噪声问题的鲁棒性学习方法层出不穷,文献\cite{zhang2021understanding,wang2018iterative}指出,尽管基于深度网络模型的分类网络具有对噪声标签的记忆能力,但在训练过程中是先学习主要的分类规则,然后才开始记忆噪声的分类模式。本章对此进行了进一步的研究,提出了一个端到端的标签噪声条件下的鲁棒性训练方法。


\subsection{标签噪声下分类方法算法框架}

% 在极化SAR图像目标分类中,标签噪声的存在通常会导致传统分类方法性能下降,基于卷积网络的模型往往也会陷入过拟合的问题,学习到错误的分类规则。标签噪声通常是由于地物杂波、系统误差、人工标记错误等因素引入,导致训练集中包含错误的标签。为了解决这个问题,本章方法提出了一种针对极化SAR的鲁棒性分类方法。图\ref{BBM_framework}展示了BBM算法的框架图。由于深度网络模型总是趋向于先学习简单样本,之后通过足够的迭代去拟合困难的样本。在标签噪声下的深度网络训练过程中,会呈现双峰的损失函数分布情况。因此,与传统的分类方法相同,本方法首先通过深度网络进行特征提取和分类器进行分类,计算每个训练样本的损失函数。得到的损失函数值并不直接参与模型的反向传播中,而是通过二分量的贝塔混合模型来拟合所有的损失函数的分布,通过损失函数的观测值来得到贝塔混合模型的参数估计。确定噪声分量的噪声概率密度函数之后,即可根据损失函数值得到对应的样本噪声概率。其次,由于极化SAR目标分布存在空间一致性,空间连续的样本往往属于同一个类别。并且,空间连续区域标注简单,标签噪声更加趋向于出现在类别边界区域。基于该特性,还设计了边界样本增强分支,对极化SAR图像进行Pauli分解,得到对应的伪彩图,在伪彩图中利用Sobel边界提取算子并通过边界膨胀,获得图像大致的膨胀边界。对于属于膨胀边界内的样本模型应该付出更多的注意,来拟合其中的特征。最后,由于训练集数量有限,如果直接丢弃属于噪声的样本,可能会导致丢失其中必要的空间信息。因此,引入了动态的鲁棒性损失函数,即赋予模型动态的根据噪声概率去选择训练集标签或者模型预测值作为交叉熵损失的计算基准。通过以上的鲁棒性训练方法,最终获得对标签噪声具备鲁棒性的分类模型。

在极化SAR图像目标分类中,标签噪声的存在通常会导致传统分类方法性能下降,基于卷积网络的模型往往也会陷入过拟合的问题,学习到错误的分类规则。针对标签噪声问题,本章提出了基于混合模型估计与边界增强的极化SAR图像分类方法,其算法框架如图\ref{BBM_framework}所示。区别于传统的交叉熵损失优化方法,本章方法基于混合模型理论对样本损失函数进行参数估计,以区分噪声样本与干净样本。为了增强模型对边界信息的感知能力,利用Sobel算子对极化SAR图像的Pauli伪彩图进行边界提取并膨胀,对属于膨胀边界内的样本进行损失加权。最后,基于改进的自学习优化损失函数,结合模型预测感知项,对噪声标签进行校正,实现模型的鲁棒性训练。

\begin{figure}[h]
    \centering
    \includegraphics[width=14cm]{pic/chapter4/framework.png}
    \caption{标签噪声下鲁棒性分类算法框架图}
    \label{BBM_framework}
\end{figure}

结合上述算法框架,本章基于混合模型估计与边界增强的极化SAR图像分类方法可以表示为以下步骤:

(1)极化特征输入。极化特征具有多种表示形式,选择合适的极化特征对于分类任务具有重要意义。本章研究属于极化SAR图像分类问题,将第三章提出的双通道注意力极化信息提取方法作为本算法的前置模块,用于极化特征表示,并输入到后续网络。

(2)鉴别特征提取与分类。利用卷积神经网络,对输入的极化特征进行特征提取。通过多个堆叠的卷积、池化与激活操作,主干网络学习到极化特征的高级表征。随后,利用全连接层与激活函数,将主干网络的输出映射到不同的类别,得到对样本的预测类别。

(3)噪声概率估计。计算每个样本预测类别与标签的交叉熵损失,并基于贝塔混合模型拟合样本损失分布。利用损失值来更新迭代混合模型参数,基于更新后的模型参数计算样本噪声概率。

(4)边界样本增强。利用Sobel算子对极化SAR图像的Pauli伪彩图进行边界提取并膨胀,对膨胀边界内部的训练样本进行损失函数加权增强,以提高模型对边界信息的学习能力。

(5)自学习损失函数优化。基于自学习损失函数,利用估计的样本噪声概率,动态调整模型预测与标签真值的权重。利用改进的自学习损失函数计算损失值后,通过反向传播更新模型参数,实现模型的训练优化。

\subsection{噪声概率估计模型}
在混合标签噪声数据集训练过程中,深度网络模型通常先学习干净样本特征,然后较长时间才能适应随机嘈杂标签。意味着训练早期,噪声样本相比于干净样本具有更高损失,可以认为训练过程中损失的分布是由干净样本分布与噪声样本分布叠加形成。因此,利用二分量混合模型来对样本损失分布进行拟合,完成样本噪声概率估计。

贝塔混合模型具有对二元概率随机变量更加灵活的拟合特性\citing{},用于拟合混合样本的损失值分布。
对于给定的样本$(x, y)$,其中$x, y, \hat{y}$分别表示PolSAR数据、相应的地面真实标签和$x$的分类器预测,交叉熵损失的标准化表示为 $\mathcal{L}_{CE}(\hat{y}, y) = \ell$。根据式\ref{eq:beta1}与式\ref{eq:beta1},混合标签噪声损失概率密度函数可以表示为:
\begin{equation}
    p(\ell)=\lambda_c \cdot p(\ell \mid \text{clean})+\lambda_n \cdot p(\ell \mid \text{noisy})
\end{equation}
其中,$p(\ell \mid \text{clean})$与$p(\ell \mid \text{noisy})$分别表示干净/噪声样本损失分布,具体表示如下:
\begin{gather}
    p(\ell \mid \text{clean})=\frac{\Gamma(\alpha_c+\beta_c)}{\Gamma(\alpha_c) \Gamma(\beta_c)} \ell^{\alpha_c-1}(1-\ell)^{\beta_c-1} \\
    p(\ell \mid \text{noisy})=\frac{\Gamma(\alpha_n+\beta_n)}{\Gamma(\alpha_n) \Gamma(\beta_n)} \ell^{\alpha_n-1}(1-\ell)^{\beta_n-1}
\end{gather}

其中,$\alpha_{c/n}, \beta_{c/n} > 0$ 表示干净/噪声样本的贝塔分布参数,$\Gamma(\cdot)$ 是伽玛函数,$\lambda_{c}$ 和 $\lambda_{n}$ 分别表示混合系数,表示两个分量的组合系数。

参数集$\Theta = \{c_{c/n}, \alpha_{c/n}, \beta_{c/n}\}$描述了整个混合分布的特征。使用最大期望(Expectation Maximization, EM)算法,实现混合模型的参数更新迭代。首先,在E步中,混合系数被固定,损失值 \(\ell\) 的后验概率 \(Q_k(\ell)=p(k \mid \ell)\) 基于贝叶斯规则进行更新,即
\begin{equation}
    Q_k(\ell)=\frac{\lambda_k p(\ell \mid \alpha_k, \beta_k)}{\sum_{j=0}^{1}\lambda_j p(\ell \mid \alpha_j, \beta_j)}
\end{equation}
其中,$j=0(1)$ 表示干净(噪声)类别。

然后,在M步中,基于更新的 \(Q_k(\ell)\),通过一种加权版本的矩方法估计 \(\alpha_k\) 和 \(\beta_k\)。

\subsection{边缘增强模块}

由于错误标记的噪声样本更多出现在类别交替的边界部分,模型需要对属于边界样本付出更多的关注度。为了加强对边界样本注意力关注和细节增强,对Pauli彩图进行Sobel算子卷积和展开运算,生成距离边界$d$以内的掩膜像素,并将其作为辅助边界损失的目标。对于输入的图像$I(x,y)$,Sobel算子使用两个$3 \times 3$的卷积和与输入图像进行卷积操作,分别计算垂直和水平垂直梯度近似。这两个卷积核如下所示:

\begin{gather}
    K_x=\left[ \begin{matrix}
            -1 & 0 & 1 \\
            -2 & 0 & 2 \\
            -1 & 0 & 1 \\
        \end{matrix} \right]
    \\
    K_y=\left[ \begin{matrix}
            -1 & -2 & -1 \\
            0  & 0  & 0  \\
            1  & 2  & 1  \\
        \end{matrix} \right]
\end{gather}
其中,$K_x$和$K_y$分别表示水平和垂直方向的卷积核。卷积操作的结果分别用$G_x$和$G_y$表示,其计算过程如下:
\begin{gather}
    G_x=I \ast K_x=\sum_{i=-1}^1{\sum_{j=-1}^1{I\left( x+i,y+j \right) \cdot K_x\left( i,j \right)}}
    \\
    G_y=I \ast K_y=\sum_{i=-1}^1{\sum_{j=-1}^1{I\left( x+i,y+j \right) \cdot K_y\left( i,j \right)}}
\end{gather}
其中,$\ast$表示卷积操作。通过以上两个卷积操作,得到图像每个像素点的水平和垂直梯度。图像在该相处点出的梯度强度和方向可以通过下式计算:

\begin{gather}
    G=\sqrt{G_x^2+G_y^2}
    \\
    \theta=arctan(\frac{G_y}{G_x})
\end{gather}
其中,$G$表示梯度强度,$\theta$表示梯度方向。

通过Sobel算子提取到Pauli彩图边界后,为了增强对边界区域的细节感知,通过对边界以距离$d$进行展开操作,得到膨胀的掩膜边界,用于后续的辅助边界损失。

\begin{algorithm}[H]
    \KwData{Pauli图像 $I$,Sobel算子 $S_x, S_y$,边界展开距离 $d$}
    \KwResult{Sobel目标边界 $\hat{I}$}
    $ X_b\gets \left( \left| \text{Conv}\left( I,S_x \right) \right|+\left| \text{Conv}\left( I,S_y \right) \right| \right) >0 $ \;
    $X_d\gets Dilate(X_b)$ \;
    $\hat{I}\gets I\otimes X_d$ \;
    返回 $\hat{I}$ \;

    其中,$\otimes$表示逐元素乘法
    \caption{Sobel目标边界生成}
    \label{Sobel}
\end{algorithm}


\subsection{损失函数与优化}
面对标签噪声问题时,损失函数的精心选择对于引导模型学习训练过程变得极为重要。传统的交叉熵损失完全依赖于分类模型预测类别与给定的训练集标签值进行计算损失,对于训练集中样本标签存在错误时会引导模型参数拟合错误的规则,会鼓励模型学习到错误的信息。为了处理标签噪声问题,本章通过在标准交叉熵损失中引入感知项来辅助调整训练目标。具体而言,损失函数表达式为:
\begin{equation}
    \ell_B=\sum_{i=1}^{N}((1-\omega_i)y_i+\omega_iz_i)^Tlog(h_i)
\end{equation}
其中,$N$表示样本数量,$y_i$与$z_i$分别表示样本标签值与分类模型预测值,$\omega_i$在损失函数中作为权重调整对样本标签值和模型预测值之间的权衡。

如果设置权重$\omega$为固定值,并不能有效防止噪声样本的过拟合情况。本章将噪声预测概率与动态损失函数相结合,通过样本属于噪声的概率来动态调整样本标签和模型预测分类的依赖程度。结合BMM模型的噪声概率预测,通过将式$\omega$设置为式\ref{BMM_PDF}得到的噪声概率$p(\ell \mid \text{noisy})$,并且在每一个训练轮次结束后使用每个样本的交叉熵损失更新BMM模型参数,进而更新对应的噪声概率。因此干净的样本的噪声概率$1-\omega_i$接近1,更多的依赖于真值标签,而属于噪声样本的损失则由模型分类结果$z_i$和预测噪声概率来决定。

\section{实验结果与分析}
\subsection{实验参数设置}
本章实验所使用的计算环境为一台CPU为Intel Core i7-8700K和配备了NVIDIA GPU(GeForce RTX 3090, 24G)的计算机设备。操作系统采用Ubuntu 20.04 LTS。深度学习框架选择PyTorch,版本为1.9.0,同时依赖CUDA深度神经网络库(cuDNN)版本8.0.5。在科学计算方面,实验使用NumPy库,版本为1.19.5。学习率、batch-size等超参数设置与第三章实验参数相同。

实验中,使用两组真实极化SAR数据集分别是荷兰Flevoland区域和德国Oberpfaffenhofen地区数据来验证本章方法的有效性,并利用常规的性能指标总体分类准确率(OA)、各个类别分类准确率和Kappa系数对分类结果进行数值量化。同时对分类的可视化结果进行视觉评估。在给定的极化SAR标准数据集中,一部分像素是没有标签的,所以在计算分类准确率的时候只统计数据集中那些有标签的样本被正确分类的百分比,并认为该指标可以表征数据集中整体的分类性能。

实验中,随机从数据集中选择1\%的样本作为训练集,其余的99\%样本作为测试集。为了验证本章方法的有效性与优越性,实验选择了两组真实极化SAR图像数据集,分别是荷兰Flevoland地区数据和德国San Francisco地区数据。为了模拟实际上镜中出现的对称噪声标签和非对称噪声标签的两种情况,本节实验通过设置噪声转移矩阵的方法,分别在以上两个数据集中合成对应类型的标签噪声。同时,为了验证本章方法对标签噪声的鲁棒性能,在非对称噪声类型中设置不同噪声比例,比较对应的准确率指标。

为了验证本章方法的优越性能,选择了多种替代的分类方案进行比较,主要是通过与目前经典的以及优越的分类方法进行对比,包括支持向量机方法(记为SVM)\citing{}、基于Wishart分布的分类方法(记为Wishart)\citing{}和鲁棒的半监督学习方法(记为RSL)\citing{}。

% 每组实验包括验证实验与对比实验两个部分,其中验证实验为本章方法在不同噪声比例下的分类结果对比,用于展示本章方法对标签噪声的有效性;对比实验是通过与现有的经典、先进的极化SAR图像目标分类方法进行结果对比,选择了多种替代的分类方案进行比较,包括SVM\citing{}、Wishart\citing{}、CV-CNN\citing{}和鲁棒学习方法\citing{},目的是对比本章方法的优越性能。

\subsection{AIRSAR Flevoland数据实验}
\subsubsection{对称标签实验}
图\ref{fig:fle_noise_uniform}展示了10\%噪声比例下对称标签噪声的噪声转移矩阵,每个类别正确标记的概率为90\%,10\%的标签噪声均匀地分布在其他类别中。
\begin{figure}[h]
    \centering
    \includegraphics[width=10.04cm]{pic/chapter4/fle/noise_uniform.pdf}
    \caption{AIRSAR Flevoland数据10\%对称噪声转移矩阵}
    \label{fig:fle_noise_uniform}
\end{figure}

图\ref{fig:fle_res_4}展示了不同实验方法在10\%对称噪声类型下分类结果图。根据图\ref{fig:SVM}、图\ref{wishart}展示的SVM与Wishart分类器的分类结果,在大多数的类别中都存在着严重的错分情况,这可能是由于这两类分类器都无法有效地应对标签噪声的影响,错误的信息对分类规则噪声严重的负面影响,进而导致分类性能下降的现象。而图\ref{fig:CNN}展示的在交叉熵损失函数优化的CNN分类器的分类结果,可以看到小麦区域和干豆存在大量的错分情况,将小麦错分为水域,将干豆区域错分为土豆,这是因为这两类目标本身散射特性比较接近,容易造成错分,同时该分类模型并没有对标签噪声进行处理,模型过拟合学习到了错误的类别信息,导致错分的情况。图\ref{fig:RSL}展示了使用鲁棒分类方法的分类结果,可以看到大量连续错分的情况得到改善,但是在草地等区域出现了小块的错分样本,将草地错分为大麦,这可能是草地覆盖区域较小,又存在噪声的情况,导致有效的训练样本数量不足,同时RSL方法是将判断为噪声的样本直接丢弃也丢失了一部分信息,在这一块区域中的分类性能有限。图\ref{fig:BEL}与图\ref{fig:BEL_DP}均是本章提出的标签噪声下鲁棒性分类方法的分类结果图,从图中可以直观看出分类效果得到提升,错分的情况大大减少,仅有少数的错分情况,并且使用极化信息提取器后在油菜籽区域的错分情况也得到视觉上的改善,这也验证了本章方法的有效性。

\begin{figure}[ht]
    \subfloat[]{
        \label{fig:SVM}
        \includegraphics[width=4.04cm]{pic/chapter4/fle/SVM-10.png}
    }
    \subfloat[]{
        \label{fig:wishart}
        \includegraphics[width=4.04cm]{pic/chapter4/fle/Wishart-10.png}
    }
    \subfloat[]{
        \label{fig:CNN}
        \includegraphics[width=4.04cm]{pic/chapter4/fle/RSL-10.png}
    }

    \subfloat[]{
        \label{fig:RSL}
        \includegraphics[width=4.04cm]{pic/chapter4/fle/CNN-10.png}
    }
    \subfloat[]{
        \label{fig:BEL}
        \includegraphics[width=4.04cm]{pic/chapter4/fle/BEL-10.png}
    }
    \subfloat[]{
        \label{fig:BEL_DP}
        \includegraphics[width=4.04cm]{pic/chapter4/fle/BEL+DP-10.png}
    }

    \subfloat[]{
        \includegraphics[width=4.04cm]{pic/chapter4/fle/gt.png}
    }
    \subfloat[]{
        \includegraphics[width=4.04cm]{pic/chapter4/fle/label.png}
    }

    \caption{AIRSAR Flevoland地区数据对称标签噪声下分类可视化结果图。(a) SVM;(b)Wishart; (c) RSL; (d) CNN; (e) BEL; (f) BEL + DP; (g) 地面真值;(h) 颜色与类别对应关系}
    \label{fig:fle_res_4}
\end{figure}

表\ref{tab:fle_res_4}展示了不同分类方法在10\%对称标签噪声比例下的分类数值结果。SVM与Wishart分类方法由于无法有效应对标签噪声的影响,绝大多数类别的分类准确率都低于60\%,存在严重的错分的情况,总体分类准确率较低,分别为58.98\%和49.78\%。基于交叉熵损失的CNN分类器在油菜籽、干豆区域分类准确率较低,但是在其他类别上的分类准确率有了一定的提升,这可能是因为深度网络模型本身对于标签噪声的一定的鲁棒性能,会先学习特征简单的样本,再去学习错误的样本信息,总体分类准确率达到88.39\%。鲁棒性分类学习方法RSL由于对判断为噪声的样本进行了丢弃,在大多数类别上准确率取得了不错的成果,但是在建筑物区域的分类准确率较低,这可能是因为建筑物区域本身样本有限,添加噪声之后导致有效的训练样本不足,进而导致了在该类别上的分类准确率较低的情况。BEL与BEL+DP分类准确率相比于其他方法有了较大的提升,两个方法占据了所有类别和所有指标的最大值。BEL+DP方法在油菜籽、甜菜等区域中分类准确率稍低于BEL方法,不足1\%的差距,但是在建筑物、甜菜等区域中的分类准确率高于BEL方法超过1\%,验证了极化信息表征对该场景下性能提升的有效性。通过以上分析证明,本章提出的BEL方法对标签噪声的有效性与优越性。

\begin{table}[]
    \begin{tabular}{cccccccc}
        \toprule[1.5bp]
        序号                        & 类别    & SVM   & Wishart & CNN   & RSL   & BEL            & BEL + DP       \\
        \midrule[0.75bp]
        1                         & 建筑物   & 37.26 & 32.64   & 69.52 & 35.34 & 96.38          & \textbf{97.75} \\
        2                         & 油菜籽   & 31.35 & 15.58   & 78.2  & 46.64 & \textbf{92.01} & 90.27          \\
        3                         & 甜菜    & 74.17 & 9.87    & 94.56 & 99.01 & 95.67          & \textbf{97.28} \\
        4                         & 干豆    & 14.61 & 14.02   & 98.43 & 54.63 & \textbf{96.92} & 96.62          \\
        5                         & 豌豆    & 50.2  & 22.9    & 95.27 & 96.39 & \textbf{99.11} & 98.8           \\
        6                         & 森林    & 66.31 & 41.52   & 94.86 & 94.34 & 96.8           & \textbf{97.7}  \\
        7                         & 苜蓿    & 62.92 & 17.71   & 88.46 & 70.53 & 96.19          & \textbf{98}    \\
        8                         & 土豆    & 53.54 & 26.89   & 93.55 & 83.1  & 95.49          & \textbf{98.42} \\
        9                         & 裸土    & 64.25 & 11.36   & 89.34 & 95.05 & 91.02          & \textbf{99.12} \\
        10                        & 草地    & 19.64 & 17.27   & 82.86 & 60.43 & 87.65          & 91.66          \\
        11                        & 大麦    & 80.89 & 10.48   & 73.07 & 89.43 & 96.26          & 97.49          \\
        12                        & 水域    & 76.9  & 59.9    & 79.15 & 59.32 & 97.37          & \textbf{99.85} \\
        13                        & 小麦 1  & 59.13 & 40.18   & 97.77 & 91.83 & \textbf{97.71} & 97.23          \\
        14                        & 小麦 2  & 45.27 & 46.52   & 71.89 & 70.16 & \textbf{94.82} & 94.22          \\
        15                        & 小麦 3  & 75.74 & 43.69   & 93.16 & 98.85 & 97.05          & \textbf{98.06} \\
        \midrule[0.75bp]
        \multicolumn{2}{c}{OA}    & 58.98 & 49.78 & 88.39   & 89.13 & 95.85 & \textbf{96.86}                  \\
        \multicolumn{2}{c}{AA}    & 54.15 & 27.36 & 86.67   & 77.04 & 95.36 & \textbf{96.83}                  \\
        \multicolumn{2}{c}{Kappa} & 56.02 & 47.16 & 87.3    & 89.12 & 95.47 & \textbf{96.58}                  \\
        \bottomrule[1.5bp]
    \end{tabular}
    \caption{AIRSAR Flevoland地区数据对称标签噪声下分类数值结果(\%)}
    \label{tab:fle_res_4}
\end{table}

\subsubsection{非对称标签实验}
图\ref{fig:fle_noise_random}展示了10\%噪声比例下非对称标签噪声的噪声转移矩阵,每个类别正确标记的概率为90\%,10\%的标签噪声随机映射到另外的一个类别中。
\begin{figure}[ht!]
    \centering
    \includegraphics[width=10.04cm]{pic/chapter4/fle/noise_random.pdf}
    \caption{AIRSAR Flevoland数据10\%非对称噪声转移矩阵}
    \label{fig:fle_noise_random}
\end{figure}

%TODO:
图\ref{fig:fle_random}展示了10\%非对称标签噪声下不同对比方法的分类结果。图\ref{fig:SVM_random}与图\ref{fig:wishart_random}依然存在大量的错分情况,并且区域内部存在大量斑点状的错误区域。图\ref{fig:CNN_random}展示了交叉熵损失下的CNN分类结果,由于训练集中油菜籽区域部分像素被标记为小麦,导致在分类结果中也存在相同的错分情况,这表明该分类方法在错误标记训练集上过拟合,学习了错误的分类规则。图\ref{fig:RSL_random}展示了RSL方法的分类结果,。。。。图\ref{fig:BEL_random}与图\ref{fig:BEL_DP_random}展示了BEL与BEL+DP方法的分类结果,可以看到由于对称噪声标签的影响,在小麦、林地、油菜籽区域出现了一定的错分情况,但是在其他大部分区域分类结果较平滑,错分样本较少,以上结果证明了本章方法在非对称噪声标签下的有效性与优越性。

\begin{figure}[ht!]
    \subfloat[]{
        \label{fig:SVM_random}
        \includegraphics[width=4.04cm]{pic/chapter4/fle/SVM-10.png}
    }
    \subfloat[]{
        \label{fig:wishart_random}
        \includegraphics[width=4.04cm]{pic/chapter4/fle/Wishart-10.png}
    }
    \subfloat[]{
        \label{fig:CNN_random}
        \includegraphics[width=4.04cm]{pic/chapter4/fle/RSL-10.png}
    }

    \subfloat[]{
        \label{fig:RSL_random}
        \includegraphics[width=4.04cm]{pic/chapter4/fle/CNN_random.png}
    }
    \subfloat[]{
        \label{fig:BEL_random}
        \includegraphics[width=4.04cm]{pic/chapter4/fle/BEL_random.png}
    }
    \subfloat[]{
        \label{fig:BEL_DP_random}
        \includegraphics[width=4.04cm]{pic/chapter4/fle/BEL+DP_random.png}
    }

    \subfloat[]{
        \includegraphics[width=4.04cm]{pic/chapter4/fle/gt.png}
    }
    \subfloat[]{
        \includegraphics[width=4.04cm]{pic/chapter4/fle/label.png}
    }

    \caption{AIRSAR Flevoland地区数据非对称标签噪声下分类可视化结果图。(a) SVM;(b)Wishart; (c) RSL; (d) CNN; (e) BEL; (f) BEL + DP; (g) 地面真值;(h) 颜色与类别对应关系}
    \label{fig:fle_random}
\end{figure}


表\ref{tab:fle_res_random}展示了各对比方法在非对称标签噪声下的分类数值结果。SVM与Wishart作为经典的极化SAR分类方法,在非对称标签标签噪声下分类效果不理想,总体准确率分别为56.48\%与50.32\%,这是因为基于SVM与Wishart的方法分类器模型简单,无法鉴别标签噪声的错误信息。基于交叉熵损失的CNN方法虽然整体有了提升,但是在建筑物区域准确率仅为42.28\%,这是因为CNN也不具备对标签噪声的鲁棒性能,会学习到错误的分类准则。除了干豆、水域、大麦三个区域略低于RSL方法,本章方法在其他指标上均取得了最高值,尤其是建筑物区域获得了较大的提升,反映了本章提出的方法是有效且优越的。

\begin{table}[ht!]
    \caption{AIRSAR Flevoland地区数据非对称标签噪声下分类数值结果(\%)}
    \label{tab:fle_res_random}
    \begin{tabular}{cccccccc}
        \toprule[1.5bp]
        序号                        & 类别    & SVM   & Wishart & CNN   & RSL            & BEL            & BEL + DP       \\
        \midrule[0.75bp]
        1                         & 建筑物   & 31.73 & 31.56   & 42.28 & 46.24          & 91.09          & \textbf{96.16} \\
        2                         & 油菜籽   & 33.76 & 18.23   & 61.25 & 82.77          & \textbf{92.34} & 92.19          \\
        3                         & 甜菜    & 73.98 & 10.56   & 74.86 & 95.23          & 96.68          & \textbf{97.92} \\
        4                         & 干豆    & 16.23 & 12.97   & 85.41 & \textbf{96.8}  & 91.01          & 94.6           \\
        5                         & 豌豆    & 48.34 & 24.56   & 87.73 & 95.62          & 97.86          & \textbf{97.39} \\
        6                         & 森林    & 64.28 & 40.08   & 82.19 & 91.36          & 96.68          & \textbf{97.96} \\
        7                         & 苜蓿    & 60.72 & 18.23   & 79.2  & 79.8           & 94.43          & \textbf{96.04} \\
        8                         & 土豆    & 52.01 & 22.14   & 81.69 & \textbf{95.9}  & 95.68          & 94.41          \\
        9                         & 裸土    & 64.58 & 14.2    & 81.32 & 98.04          & 94.01          & \textbf{99.39} \\
        10                        & 草地    & 21.33 & 19.32   & 77.84 & 64.59          & 92.98          & \textbf{92.4}  \\
        11                        & 大麦    & 78.89 & 9.47    & 76.23 & 59.39          & 89.52          & \textbf{97.24} \\
        12                        & 水域    & 75.42 & 58.14   & 68.04 & \textbf{98.2}  & 97.35          & 97.7           \\
        13                        & 小麦 1  & 57.6  & 42.18   & 80.37 & \textbf{94.83} & 93.97          & 93.87          \\
        14                        & 小麦 2  & 44.32 & 44.25   & 65.17 & 86.95          & 83.83          & \textbf{91.55} \\
        15                        & 小麦 3  & 74.36 & 45.84   & 70.35 & 94.2           & 94.72          & \textbf{98.12} \\
        \midrule[0.75bp]
        \multicolumn{2}{c}{OA}    & 56.48 & 50.32 & 75.28   & 89.88 & 93.82          & \textbf{96.52}                  \\
        \multicolumn{2}{c}{AA}    & 53.17 & 27.45 & 74.26   & 88.91 & 91.48          & \textbf{95.79}                  \\
        \multicolumn{2}{c}{Kappa} & 54.59 & 48.03 & 73.96   & 88.94 & 93.26          & \textbf{95.22}                  \\
        \bottomrule[1.5bp]
    \end{tabular}
\end{table}

图\ref{fig:fle_noise}展示了两种噪声源下不同噪声比例的各个方法在AIRSAR Flevoland地区数据中的分类总体准确率。从图中可以看到,随着噪声比例的增加,所有分类方法的准确率都有所下降,只是下降的幅度具有差异。SVM、Wishart、CNN这三种未考虑噪声标签影响的分类方法随着噪声比例增加下降幅度较快,从10\%噪声比例至50\%分别下降了22.70\%、13.91\%和18.40\%,并且还是在方法的分类准确率不高的情况下,表明了两种类型的标签噪声带来的错误信息对分类任务形成了较大的影响。对于RSL方法,从10\%噪声比例至50\%分类准确率下降了10.59\%,下降幅度相对较小,但是在50\%噪声比例下分类准确率低于80\%,表明该方法对高比例噪声标签的鲁棒性能有限。BEL与BEL+DP方法从10\%噪声比例至50\%分类准确率分别下降了9.9\%和8.69\%,准确率下降幅度均小于10\%,且在50\%噪声比例下准确率依然均在85\%以上,这验证了本章方法的有效性。
\begin{figure}[ht!]
    \subfloat[]{
        \includegraphics[width=7.04cm]{pic/chapter4/fle/noise_sy.pdf}
    }
    \subfloat[]{
        \includegraphics[width=8.04cm]{pic/chapter4/fle/noise_in.pdf}
    }
    \caption{AIRSAR Flevoland地区数据不同噪声比例下分类准确率结果。(a) 对称噪声源; (b) 非对称噪声源}
    \label{fig:fle_noise}
\end{figure}

\subsection{E-SAR Oberpfaffenhofen数据实验}
\subsubsection{对称标签实验}
图\ref{fig:ober_noise_uniform}展示了10\%噪声比例下对称标签噪声的噪声转移矩阵,每个类别正确标记的概率为90\%,10\%的标签噪声均匀地分布在其他类别中。
\begin{figure}[ht!]
    \centering
    \includegraphics[width=10.04cm]{pic/chapter4/ober/noise_uniform.pdf}
    \caption{E-SAR Oberpfaffenhofen数据10\%对称噪声转移矩阵}
    \label{fig:ober_noise_uniform}
\end{figure}

图\ref{fig:ober_res_random}展示了不同分类方法在10\%对称标签噪声条件下的分类可视化图。根据图\ref{fig:ober_Wishart}与\ref{fig:ober_SVM}展示的Wishart与SVM的分类结果,可以看出分类效果较差,尤其是在林地、建筑区域存在大量的错分样本,将建筑物区域错分为开放区,将林地区域错分为建筑区域,一方面是因为Wishart与SVM分类方法的分类性能有限,对散射特性复杂的极化SAR数据分类结果不理想,另一方面是因为将错误的标签噪声信息归为学习规则之一,学习到错误的分类规则,导致分类性能的进一步下降。根据图\ref{fig:ober_CNN}展示的在交叉熵损失函数优化的CNN分类器的分类结果,可以看到相比于SVM与Wishart的分类效果有了一定的提升,但是依然存在着错误分类情况,主要是将建筑物区域分类为开放区,这是因为建筑物区域的散射特性差异大,并且交叉上损失函数的CNN会学习错误的标签信息,造成在训练集上过拟合,进而导致在测试集错分的情况。根据图\ref{fig:ober_RSL}展示的RSL方法的分类结果,经过对噪声标签概率估计后过滤可能是错误的信息的操作,可以看到分类效果直观好于CNN方法,但是依然在建筑区、林地区域存在少量的错误块,这可能是因为RSL直接丢弃了错误的样本,丢失了训练样本中的数据分布特征,导致分类性能有限。图\ref{fig:ober_BEL}与图\ref{fig:ober_BEL_DP}展示的BEL与BEL+DP方法的分类结果,可以看到错分情况得到改善,在类别区域内部更加平滑,错分块数量减少,证明了本章方法的有效性与优越性。
\begin{figure}[ht!]
    \subfloat[]{
        \label{fig:ober_Wishart_}
        \includegraphics[width=4.04cm]{pic/chapter4/ober/Wishart-10.png}
    }
    \subfloat[]{
        \label{fig:ober_SVM}
        \includegraphics[width=4.04cm]{pic/chapter4/ober/SVM-10.png}
    }
    \subfloat[]{
        \label{fig:ober_CNN}
        \includegraphics[width=4.04cm]{pic/chapter4/ober/CNN-10.png}
    }

    \subfloat[]{
        \label{fig:ober_RSL}
        \includegraphics[width=4.04cm]{pic/chapter4/ober/RSL-10.png}
    }
    \subfloat[]{
        \label{fig:ober_BEL}
        \includegraphics[width=4.04cm]{pic/chapter4/ober/BEL-10.png}
    }
    \subfloat[]{
        \label{fig:ober_BEL_DP}
        \includegraphics[width=4.04cm]{pic/chapter4/ober/BEL_DP-10.png}
    }

    \subfloat[]{
        \includegraphics[width=4.04cm]{pic/chapter4/ober/gt.png}
    }
    \subfloat[]{
        \includegraphics[width=2.04cm]{pic/chapter4/ober/label.png}
    }

    \caption{AIRSAR Flevoland地区数据对称标签噪声下分类可视化结果图。(a) SVM; (b) Wishart; (c) RSL; (d) CNN; (e) BEL; (f) BEL + DP; (g) 地面真值;(h) 颜色与类别对应关系}
    \label{fig:ober_res_random}
\end{figure}

表\ref{tab:ober_res_4}展示了不同实验方法在10\%比例对称标签噪声下的分类数值结果。其中,SVM与Wishart分类方法虽然在开放区域的分类准确率取得了不错的效果,分别为95.28\%和95.21\%,但是由于在建筑区域和林地区域分类效果均不理想,导致整体准确率、平均准确率、Kappa系数三相指标都较低,这是因为噪声标签信息对该两个分类方法的影响,学习到错误的分类规则。RSL方法在林地区域单向准确率取得了最高的分类准确率为95.16\%,相比于本章方法提高了3.94\%,这可能是因为林地区域散射特性差异大,极化信息提取器在该区域的提升性能不足,同时该区域训练样本充足,直接丢弃所有的可能为噪声的样本并未丢失关键的数据分布特征。RSL+DP方法在综合指标方面均取得了最高值,分别为94.55\%、93.52\%和90.66\%,相比于其他方法分别提升了4.76\%、5.6\%和1.23\%。综上所述,本章方法在噪声标签条件下的数据集中具有更优越的分类性能。

\begin{table}[ht!]
    \caption{ESAR Oberpfaffenhofen地区数据对称标签噪声下分类数值结果(\%)}
    \label{tab:ober_res_4}
    \begin{tabular}{cccccccc}
        \toprule[1.5bp]
        序号                        & 类别    & SVM   & Wishart & CNN   & RSL            & BEL            & BEL+DP         \\
        \midrule[0.75bp]
        1                         & 建筑    & 75.19 & 61.35   & 81.25 & 75.32          & 89.98          & \textbf{92.35} \\
        2                         & 林地    & 76.54 & 74.58   & 80.24 & \textbf{95.16} & 91.59          & 91.82          \\
        3                         & 开放区域  & 95.28 & 95.21   & 85.82 & 93.28          & 93.39          & \textbf{96.38} \\
        \midrule[0.75bp]
        \multicolumn{2}{c}{OA}    & 79.07 & 83.05 & 83.67   & 89.79 & 92.56          & \textbf{94.55}                  \\
        \multicolumn{2}{c}{AA}    & 82.33 & 77.04 & 82.44   & 87.92 & 91.65          & \textbf{93.52}                  \\
        \multicolumn{2}{c}{Kappa} & 77.25 & 70.45 & 79.14   & 89.43 & 89.68          & \textbf{90.66}                  \\
        \bottomrule[0.75bp]
    \end{tabular}
\end{table}

\subsubsection{非对称标签实验}
图\ref{fig:ober_noise_random}展示了10\%噪声比例下非对称标签噪声的噪声转移矩阵,每个类别正确标记的概率为90\%,10\%的标签噪声随机映射到另外的一个类别中。
\begin{figure}[ht!]
    \centering
    \includegraphics[width=10.04cm]{pic/chapter4/ober/noise_random.pdf}
    \caption{ESAR Oberpfaffenhofen数据10\%非对称噪声转移矩阵}
    \label{fig:ober_noise_random}
\end{figure}

图\ref{fig:fle_res_random}展示了10\%非对称标签噪声下不同分类方法的分类结果。图\ref{fig:ober_Wishart_random}与图\ref{fig:ober_SVM_random}展示的Wishart与SVM的分类结果,可以在林地、建筑区域都存在较多的错误区域,分类效果差。图\ref{fig:ober_CNN_random}为交叉熵损失下的CNN分类结果,可以看到受到标签噪声影响,开放区域错分为建筑、建筑区域错分为林地的情况存在较多。图\ref{fig:ober_RSL_random}为RSL方法的分类结果。。。。图\ref{fig:ober_BEL_random}与图\ref{fig:ober_BEL_DP_random}为BEL与BEL+DP方法的分类结果,可以看出在建筑物区域和林地区域的错分情况减少,分类结果更加平滑。因此,证明了本章方法的有效性与优越性。
\begin{figure}[ht!]
    \subfloat[]{
        \label{fig:ober_Wishart_random}
        \includegraphics[width=4.04cm]{pic/chapter4/ober/Wishart-10.png}
    }
    \subfloat[]{
        \label{fig:ober_SVM_random}
        \includegraphics[width=4.04cm]{pic/chapter4/ober/SVM-10.png}
    }
    \subfloat[]{
        \label{fig:ober_CNN_random}
        \includegraphics[width=4.04cm]{pic/chapter4/ober/CNN-10-random.png}
    }

    \subfloat[]{
        \label{fig:ober_RSL_random}
        \includegraphics[width=4.04cm]{pic/chapter4/ober/RSL-10.png}
    }
    \subfloat[]{
        \label{fig:ober_BEL_random}
        \includegraphics[width=4.04cm]{pic/chapter4/ober/BEL-10-random.png}
    }
    \subfloat[]{
        \label{fig:ober_BEL_DP_random}
        \includegraphics[width=4.04cm]{pic/chapter4/ober/BEL_DP-random.png}
    }

    \subfloat[]{
        \includegraphics[width=4.04cm]{pic/chapter4/ober/gt.png}
    }
    \subfloat[]{
        \includegraphics[width=2.04cm]{pic/chapter4/ober/label.png}
    }

    \caption{AIRSAR Flevoland地区数据非对称标签噪声下分类可视化结果图。(a) SVM; (b) Wishart; (c) RSL; (d) CNN; (e) BEL; (f) BEL + DP; (g) 地面真值;(h) 颜色与类别对应关系}
    \label{fig:fle_res_random}
\end{figure}

表\ref{tab:ober_res_random}展示了非对称标签噪声下的分类数值结果。SVM、Wishart以及CNN方法在建筑区域的分类准确率较低,导致总体准确率较低,分别为76.58\%,80.32\%,82.49\%,这是因为这三个方法均没有筛选错误样本的能力,学习错误的分类规则。本章方法除了林地区域,其他的分类指标均取得最高值,分类准确率均在90\%以上,且总体分类准确率达到94.32\%,较其他方法提升了2.74\%,这也证明了本章方法具备应对非对称标签噪声的鲁棒性,进一步证明本章方法的有效性。
\begin{table}[ht!]
    \caption{ESAR Oberpfaffenhofen地区数据非对称标签噪声下分类数值结果(\%)}
    \label{tab:ober_res_random}
    \begin{tabular}{cccccccc}
        \toprule[1.5bp]
        序号                        & 类别    & SVM   & Wishart & CNN   & RSL            & BEL            & BEL+DP         \\
        \midrule[0.75bp]
        1                         & 建筑    & 72.38 & 58.83   & 78.92 & 86.73          & 88.06          & \textbf{90.48} \\
        2                         & 林地    & 74.54 & 72.43   & 81.32 & \textbf{94.44} & 89.44          & 90.83          \\
        3                         & 开放区域  & 92.28 & 92.31   & 83.16 & 90.53          & 94.04          & \textbf{97.18} \\
        \midrule[0.75bp]
        \multicolumn{2}{c}{OA}    & 76.58 & 80.32 & 82.49   & 91.58 & 91.8           & \textbf{94.32}                  \\
        \multicolumn{2}{c}{AA}    & 79.73 & 74.52 & 81.13   & 90.56 & 90.51          & \textbf{92.83}                  \\
        \multicolumn{2}{c}{Kappa} & 74.71 & 68.93 & 77.32   & 86.97 & 87.88          & \textbf{90.31}                  \\
        \bottomrule[0.75bp]
    \end{tabular}
\end{table}

图\ref{fig:ober_noise}展示了两种噪声源下不同噪声比例的各个方法在ESAR Oberpfaffenhofen地区数据中的分类总体准确率。可以看出,各个实验方法的分类准确率随着标签噪声比例的增加而逐渐减少。显然,SVM、Wishart、CNN这三种方法由于未考虑噪声标签的错误信息,在噪声比例较高时分类规则明显被改变,导致分类准确率大幅度下降,从10\%的噪声比例至50\%分别下降了20.64\%、28.82\%和15.30\%。而考虑了标签噪声的RSL方法分类准确率下降了13.29\%,表现出对标签噪声具有一定的鲁棒性,但是在50\%噪声比例下准确率仅为76.5\%,也表现出了该方法的局限性。而BEL与BEL+DP方法分别下降了7.62\%和7.8\%,展现出对标签噪声的优越的鲁棒性能,同时在50\%噪声比例下依然保持在85\%左右及以上,验证了本章方法在该数据集上对标签噪声的有效性与优越性。
\begin{figure}[ht!]
    \subfloat[]{
        \includegraphics[width=7.04cm]{pic/chapter4/ober/noise_sy.pdf}
    }
    \subfloat[]{
        \includegraphics[width=8.04cm]{pic/chapter4/ober/noise_in.pdf}
    }
    \caption{ESAR Oberpfaffenhofen地区数据不同噪声比例下分类准确率结果。(a) 对称噪声源 (b) 非对称噪声源}
    \label{fig:ober_noise}
\end{figure}

\subsection{本章小结}
针对极化SAR图像中存在标签噪声污染问题,本章研究了基于混合模型和边界增强的鲁棒性极化SAR目标分类方法。首先介绍了有限混合模型,并分析了两种混合模型的优缺点。在此基础上,通过深度学习在包含标签噪声样本下训练损失分布存在差异,提出了基于混合模型的样本概率估计方法。同时,为了增强模型对边界信息的感知能力,充分利用边界样本信息,对极化SAR的Pauli伪彩图使用Sobel算子提取边界并膨胀,对膨胀边界内样本进行损失增强。最后,提出了动态的自学习损失函数,通过噪声样本概率动态调整模型预测与真值标签的依赖,实现模型的鲁棒性训练过程。最终,本章通过在两个真实极化SARS数据集,对两种噪声源进行模拟实验,验证本章方法的有效性与优越性。同时,还对不同噪声比例下各个方法的分类准确率进行了比较,进一步验证本章方法的有效性与优越性。
\chapter{全文总结与展望}
\section{全文总结}
极化合成孔径雷达通过收发不同极化方式的电磁波对地物目标进行探测,提供了丰富的目标散射信息,广泛应用于军事、民事等领域中。极化SAR图像解译作为极化SAR应用的关键步骤,是当前研究的重点方向。本文以极化SAR图像目标分类为主要研究问题,展开了对极化特征表示与标签噪声目标分类的研究。极化SAR图像蕴含丰富的极化信息,不同类型的极化信息从不同角度描述地物目标散射特性,本文提出了双通道注意力的极化SAR目标分类,对散射特征和目标分解特征进行联合,提升极化信息的可表达性,提高极化SAR目标分类准确率。针对极化SAR图像分类中标签噪声问题,本文提出了基于混合模型估计的鲁棒目标分类方法,通过对样本损失分布建模,实现噪声样本的概率估计,通过鲁棒性分类方法联合极化信息提取器,在多组实验数据中证实有效提升了分类精度。具体的研究内容如下:

(1)将极化SAR图像中常用的散射特征与目标分解特征相结合,设计了双通道注意力的极化SAR目标分类方法。该方法利用双通道的网络结构,基于注意力方法,实现对散射特征和目标分解特征中关键极化信息的捕获。利用联合注意力调整方法,实现对两类注意力参数的调整,结合不同类型的极化信息。基于多尺度学习方法,增强对不同空间尺寸的极化信息的感知。最后在两个不同的极化SAR数据集上进行验证对比。

(2)在上一个工作基础上,针对极化SAR图像分类中的标签噪声问题,提出了基于混合模型估计的鲁棒性分类方法。该方法首先通过混合模型对样本损失分布进行拟合,不断更新迭代参数,利用模型参数估计样本噪声概率。同时,为了提升模型对边界信息的利用,增强对边界重要样本的感知能力,通过边界提取与膨胀操作,对膨胀边界内的训练样本进行损失增强。最后,改进自学习损失函数优化方法,通过噪声概率实现对模型预测与样本标签的依赖调整,实现鲁棒训练过程,最终提升分类精度。在两个不同的数据集上,分别验证了对称噪声与非对称噪声下的准确率精度。

\section{后续工作展望}
极化SAR图像信息提取与目标分类研究是目前极化SAR图像解译的关键步骤,创新方法层出不穷,一直是研究的热点与难点。

极化SAR图像信息提取与目标分类研究是一个设计多个跨学科交叉领域的遥感图像处理技术,包括雷达成像、微波遥感、模式识别和机器学习等。相关研究的创新方法层出不穷,一直都是研究的热点与难点。本文针对极化信息提取与标签噪声目标分类问题展开研究,提出一些新的解决思路与方案,但是依然存在许多问题亟待解决,有待在将来的工作中进一步完善,主要包括:

(1)极化特征利用方面:

本文第三章针对极化SAR图像的散射特征和目标分解特征的综合利用展开了研究,旨在对不同类型的极化特征进行有效联合,利用双通道网络结构与注意力调整方法实现独立特征与联合特征的捕捉。该方法尽管考虑了对散射特征与目标分解特征的差异与联系,而极化SAR图像中还存在空间特征、颜色特征、纹理特征等。因此,设计更加全面的极化信息提取器,构建全面、高效的极化信息融合方法,对于极化SAR图像解译具有支撑性作用。

(2)目标分类方面:

本文提出的极化SAR目标分类方法是以单个像素为单位,而通常极化SAR数据集数量巨大,逐个对极化SAR像素进行分类可能会导致计算量大、计算时间长的问题。因此,在后面的工作中,可以在减小时间复杂度的方面考虑,设计更加快速、准确的极化SAR图像分类方法。

(3)噪声标签分类方面:

本文第四章通过混合模型与边界增强,实现了对标签噪声具有鲁棒性的分类方法。但是该方法的分类损失模型仅是通过调整预测标签与真值标签的权衡,并没有完全消除标签噪声被学习的过程。因此,在明确标签噪声学习理论的基础上,构建更加高效的目标损失函数,对于提升标签噪声样本学习模型的性能具有重要意义。




% misc

\thesisacknowledgement
感谢党和国家

% \input{misc/appendix}



%
% Uncomment the following code to load bibliography database with native
% \bibliography command.
%
% \nocite{*}
\bibliographystyle{thesis-uestc}
\bibliography{reference}
%

% \thesisaccomplish{publications}
% \input{misc/translate_original}
% \input{misc/translate_chinese}

\begin{thesistheaccomplish}
    \section*{已发表或录用的论文:}
    \bibitem{111} \textbf{Lin X}, Ma Y, et al. Beta Mixture Model And Boundary Amplification Guided Label Noise Mitigation For PolSAR Image Classification[C]. IEEE International Geoscience and Remote Sensing Symposium 2024 (IGARSS 2024). (已录用)
    \section*{已申请的国家发明专利:}

\end{thesistheaccomplish}

\end{document}
