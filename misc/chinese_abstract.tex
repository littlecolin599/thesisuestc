\begin{chineseabstract}
    合成孔径雷达(Synthetic Aperture Radar, SAR)是一种对地成像雷达,具有全天时、全天候对地物目标进行持续探测的特性。极化SAR在SAR的基础上,通过发射、接收不同极化方式的电磁波信号,实现对地物目标的极化散射特性探测,具有更丰富的极化信息。极化SAR图像解译技术目前已经广泛应用于军事探测、灾害分析、城市规划等多个领域。

    随着极化SAR技术的不断发展,日益增多的高分辨、多极化的SAR图像数据为复杂场景分析提供了更丰富的地物极化信息。虽然丰富的极化信息、增加的数据量为极化SAR解译带来积极作用,但是也带来了新的挑战。丰富的极化信息要求特征表示方法能有效利用不同类型极化信息。此外,极化SAR数据集标注过程中引入的标签噪声要求分类方法能在部分错误标记样本下进行准确的分类。本章针对极化信息提取与标签噪声问题,提出了适用于极化SAR的信息提取方法与鲁棒的目标分类方法。

    本文的主要研究工作如下:

    (1)开展标签噪声下极化SAR图像分类研究之前,一个性能优越的极化信息提取方法是有必要的。针对多类型极化信息表征问题,提出了一种基于双通道注意力的极化信息提取方法。该方法基于极化SAR散射特征与目标分解特征的差异性,构建双通道的网络结构,利用注意力方法分别对两通道内极化信息进行细化。在此基础上,基于两类极化特征的差异性与互补性,设计极化注意力调整模块,对两类型极化信息进行整合。此外,利用跨空间特征学习方法,聚合不同空间尺度的极化特征。实验证明,相比于使用单类型极化特征或直接叠加极化特征,本方法提升了分类任务准确度,为极化SAR数据提供了一种更全面、充分的表达方式。

    (2)在上述研究基础上,针对极化SAR标签噪声问题,提出了基于混合模型估计与边界增强的鲁棒性分类方法。该方法根据噪声、干净样本在深度网络模型中损失的分布差异,基于贝塔混合模型对损失分布进行拟合,实现样本噪声概率估计。同时,基于Sobel算子对极化SAR的Pauli伪彩图进行边界提取并膨胀,对膨胀边界内部的样本增加额外惩罚。最后,对自学习损失函数改进,通过样本噪声概率动态调整预测与标签之间的权衡,实现鲁棒性的学习优化。实验证明,本方法对不同类型噪声源、不同噪声比例均有更优越的鲁棒性能以及更优异的分类精度。

    % 本文研究工作
    % 总结


    \chinesekeyword{极化合成孔径雷达,极化SAR图像,信息提取,目标分类,标签噪声}
\end{chineseabstract}

