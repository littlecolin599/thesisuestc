\begin{chineseabstract}
    % 合成孔径雷达(Synthetic Aperture Radar, SAR)是一种对地成像雷达,具有全天时、全天候对地物目标进行持续探测的特性。极化SAR在SAR的基础上,通过发射、接收不同极化方式的电磁波信号,实现对地物目标的极化散射特性探测,具有更丰富的极化信息。极化SAR图像解译技术目前已经广泛应用于军事探测、灾害分析、城市规划等多个领域。

    % 极化合成孔径雷达(Polarimetric Synthetic Aperture Radar, PolSAR)是一种主动式微波遥感观测技术,具有全天时、全天候的成像特点。由于其多通道、多极化的工作特性,极化SAR图像蕴含丰富的目标散射信息。极化SAR图像解译技术目前已经广泛应用于军事探测、灾害分析、城市规划等多个领域。

    % 近年来,高分辨、多极化的SAR图像数据日益增多,为复杂场景分析提供了更丰富的地物目标极化信息,对极化SAR解译工作起到了积极作用,但也带来了新的挑战。直接堆叠所有类型的极化信息存在信息冗余,而单一类型极化信息又无法适用于所有分类目标,这要求特征表示方法能有效利用多类型极化信息。此外,极化SAR数据集标注过程中引入的标签噪声要求分类方法能在部分错误标记样本下进行准确的分类。本文针对以上问题,开展了相关基础研究与创新,主要研究内容如下:

    % (1)研究了极化SAR相关理论基础。首先,介绍了电磁波极化特性的表征方式。在此基础上,介绍了目标散射特性的几种描述方法以及几种极化矩阵。最后,引出了极化目标分解方法的相关理论阐述,并总结归纳了不同极化参数的物理含义。

    % (2)针对多类型极化特征信息冗余问题,提出了一种基于双通道注意力的极化SAR目标分类方法。通过构建双通道的极化特征输入网络结构,利用空间、通道注意力模块和多尺度学习方法,实现不同类型极化特征细化与互信息提取,并聚合不同尺度的极化特征,既避免了单一特征的局限性,又减少了多特征信息冗余,提升了极化SAR目标分类准确率。

    % (3)针对标签噪声问题,提出了一种标签噪声下鲁棒的极化SAR目标分类方法。通过建立深度网络模型下噪声、干净样本损失分布差异关系,获取样本噪声概率估计,结合边界样本损失增强,实现标签噪声下鲁棒的参数优化过程,同时对边界信息充分利用,提升标签噪声下极化SAR目标分类准确率。

    % 利用真实极化SAR数据对本文提出的方法进行了有效性与优越性验证。结果显示,这些方法能够有效地应对极化SAR图像目标分类中存在的多极化特征信息冗余和标签噪声问题,实现地物目标的高准确率分类。


    % 近年来,深度学习方法取得了显著成果,为进一步提升分类质量提供了新的思路。

    % 具有全天时、全天候的成像特点。极化SAR图像分类方法作为解读极化SAR数据的核心技术手段之一,在军事应用及民用范畴内均展现出显著的应用潜力与重要价值。

    % 由于其多通道、多极化的工作特性,极化SAR图像中蕴含了丰富的目标散射信息。

    %另一方面,不同于光学图像,由于SAR独特的成像机制,极化SAR图像会不可避免的受到乘性噪声影响,图像中目标边缘及内部会处存在类别模糊和特征信息不可靠的问题。因此,考虑到SAR图像特性,研究对标签噪声鲁邦的高性能分类模型,是极化SAR图像分类技术满足实际应用需求的关键问
    极化合成孔径雷达(Polarimetric Synthetic Aperture Radar, PolSAR)凭借其全天候、全天时、多极化的成像特点而备受瞩目。极化SAR图像分类方法是解析极化SAR数据的关键技术之一,在军事和民用领域展现出重要的实用价值。

    随着极化SAR技术的演进,高分辨、多极化的SAR图像数据日益增多,对极化SAR图像分类起到了积极作用,但也引入了新的挑战。一方面,单一极化特征不足以全面满足多元化地物分类需求,而多极化特征间的冗余信息反而影响分类精度,这要求目标分类方法能有效融合利用多极化特征。另一方面,极化SAR图像标注过程难以避免的样本错误标记导致分类性能下降,这要求目标分类方法具备标签噪声鲁棒能力。针对以上问题,本文基于深度学习方法展开了相关基础研究与方法创新,主要研究内容如下:

    (1)研究了极化SAR相关理论基础。首先,介绍了电磁波极化特性的常用数学表征方式。然后,导出描述目标散射特性的极化散射矩阵和二阶统计矩阵两种方法。最后,总结了极化相干与非相干目标分解原理及代表方法,并归纳了不同极化参数的物理含义,为后续研究建立理论基础。

    (2)针对多类型极化特征信息冗余导致的分类精度下降问题,提出了基于双通道注意力的极化SAR图像分类方法。通过构建双通道、多尺度特征融合结构,利用极化SAR图像不同极化类型、不同空间位置以及不同特征尺度下的特征相关性,实现了对多类型极化特征的互信息提取和融合利用,并在此基础上构建深度分类模型,有效提高了极化SAR图像的分类准确率。

    (3)针对标签噪声导致的分类性能下降问题,提出了基于混合模型噪声估计的极化SAR图像分类方法。基于有限混合模型理论,建立了噪声标签和准确标签混杂下的损失分布模型,获取极化SAR图像中噪声标签的分布估计。此外,通过结合极化伪彩图边缘信息提取方法,使分类模型利用边缘知识增强识别性能,有效提升了标签噪声下的极化SAR图像分类准确率。

    利用实测极化SAR数据对本文提出的方法进行验证。结果表明,本文所提方法能够有效地应对极化SAR图像目标分类中存在的多极化特征信息融合利用和标签噪声下的稳健识别问题,有效提高了极化SAR图像的分类准确率。

    \chinesekeyword{极化合成孔径雷达,图像分类,深度学习,多极化特征,标签噪声}

\end{chineseabstract}