\begin{chineseabstract}
  % 合成孔径雷达(Synthetic Aperture Radar, SAR)是一种对地成像雷达,具有全天时、全天候对地物目标进行持续探测的特性。极化SAR在SAR的基础上,通过发射、接收不同极化方式的电磁波信号,实现对地物目标的极化散射特性探测,具有更丰富的极化信息。极化SAR图像解译技术目前已经广泛应用于军事探测、灾害分析、城市规划等多个领域。

  极化合成孔径雷达(Polarimetric Synthetic Aperture Radar, PolSAR)是一种主动式微波遥感观测技术,具有全天时、全天候的成像特点。由于其多通道、多极化的工作特性,极化SAR图像蕴含丰富的目标散射信息。极化SAR图像解译技术目前已经广泛应用于军事探测、灾害分析、城市规划等多个领域。

  近年来,高分辨、多极化的SAR图像数据日益增多,为复杂场景分析提供了更丰富的地物目标极化信息,对极化SAR解译工作起到了积极作用,但也带来了新的挑战。直接堆叠所有类型的极化信息存在信息冗余,而单一类型极化信息又无法适用于所有分类目标,这要求特征表示方法能有效利用多类型极化信息。此外,极化SAR数据集标注过程中引入的标签噪声要求分类方法能在部分错误标记样本下进行准确的分类。本文针对以上问题,开展了相关基础研究与创新,主要研究内容如下:

  (1)研究了极化SAR相关理论基础。首先,介绍了电磁波极化特性的表征方式。在此基础上,介绍了目标散射特性的几种描述方法以及几种极化矩阵。最后,引出了极化目标分解方法的相关理论阐述,并总结归纳了不同极化参数的物理含义。

  (2)针对多类型极化特征信息冗余问题,提出了一种基于双通道注意力的极化SAR目标分类方法。通过构建双通道的极化特征输入网络结构,利用空间、通道注意力模块和多尺度学习方法,实现不同类型极化特征细化与互信息提取,并聚合不同尺度的极化特征,既避免了单一特征的局限性,又减少了多特征信息冗余,提升了极化SAR目标分类准确率。

  (3)针对标签噪声问题,提出了一种标签噪声下鲁棒的极化SAR目标分类方法。通过建立深度网络模型下噪声、干净样本损失分布差异关系,获取样本噪声概率估计,结合边界样本损失增强,实现标签噪声下鲁棒的参数优化过程,同时对边界信息充分利用,提升标签噪声下极化SAR目标分类准确率。

  利用真实极化SAR数据对本文提出的方法进行了有效性与优越性验证。结果显示,这些方法能够有效地应对极化SAR图像目标分类中存在的多极化特征信息冗余和标签噪声问题,实现地物目标的高准确率分类。

  \chinesekeyword{极化合成孔径雷达,极化SAR图像,极化特征,目标分类,标签噪声}

\end{chineseabstract}