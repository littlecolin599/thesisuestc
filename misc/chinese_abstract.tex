\begin{chineseabstract}
    合成孔径雷达(Synthetic Aperture Radar, SAR)是一种对地成像雷达,具有全天时、全天候对地物目标进行持续探测的特性。极化SAR在SAR的基础上,通过发射、接收不同极化方式的电磁波信号,实现对地物目标的极化散射特性探测,具有更丰富的极化信息。极化SAR图像解译技术目前已经广泛应用于军事探测、灾害分析、城市规划等多个领域。

    随着极化SAR技术的不断发展,高分辨、多极化的SAR图像数据日益增多,为复杂场景分析提供了更丰富的地物极化信息。丰富的极化信息、增加的数据量为极化SAR解译带来了积极作用,但是也带来了新的挑战。丰富的极化信息要求特征表示方法能高效联合不同类型极化信息。此外,极化SAR数据集标注过程中引入的标签噪声要求分类方法能在存在错误标记下进行精确的分类。本章针对极化信息提取与标签噪声问题,提出了适用于极化SAR的信息提取方法与鲁棒的目标分类方法。

    本文的主要研究工作如下:

    (1)开展标签噪声下极化SAR图像分类研究之前,一个性能优越的极化信息提取方法是有必要的。针对多类型极化信息表征问题,提出了一种基于双通道注意力的极化信息提取方法。该方法根据极化SAR散射特征与目标分解特征的差异性,利用双通道注意力的网络结构,独立提取极化关键信息。在此基础上通过信息联合,对双通道的两种类型极化信息进行整合,挖掘两类特征间的互补信息。此外,利用跨空间特征学习方法,聚合不同尺度极化特征。联合以上方法,实现地物目标极化信息更好的表征,从而提升下游分类任务精度。实验证明,相比于使用单类型极化特征或直接叠加极化特征,本方法提升了分类任务准确度。

    (2)在上述研究基础上,针对极化SAR标签噪声问题,提出了基于混合模型估计与边界增强的鲁棒性分类方法。该方法根据噪声样本、干净样本在深度网络模型中损失的分布差异,基于贝塔混合模型对损失分布进行拟合,实现样本噪声概率估计。同时,基于Sobel算子对极化SAR的Pauli伪彩图进行边界提取并膨胀,对膨胀边界内部的样本进行损失增强。最后,对自学习损失函数改进,通过样本噪声概率动态调整预测与标签之间的权衡,实现鲁棒性的学习优化。

    本文提出的方法通过了真实极化SAR数据集验证,实验结果验证了这些方法在极化信息提取与标签噪声下目标分类的有效性,提升地物目标分类任务准确率。



    % 本文研究工作
    % 总结


    \chinesekeyword{极化合成孔径雷达,极化SAR图像,极化信息提取,地物目标分类,标签噪声}
\end{chineseabstract}

