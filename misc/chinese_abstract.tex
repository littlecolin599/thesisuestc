\begin{chineseabstract}
  合成孔径雷达(Synthetic Aperture Radar, SAR)是一种对地成像雷达,具有全天时、全天候对地物目标进行持续探测的特性。极化SAR在SAR的基础上,通过发射、接收不同极化方式的电磁波信号,实现对地物目标的极化散射特性探测,具有更丰富的极化信息。极化SAR图像解译技术目前已经广泛应用于军事探测、灾害分析、城市规划等多个领域。

  日益增多的高分辨、多极化的SAR图像数据为复杂场景分析提供了更丰富的地物目标极化信息,为极化SAR解译工作带来了积极作用,但带来了新的挑战。直接堆叠所有类型的极化信息存在信息冗余,而单一类型极化信息又无法适用于所有分类目标,这要求特征表示方法能有效利用多类型极化信息。此外,极化SAR数据集标注过程中引入的标签噪声要求分类方法能在部分错误标记样本下进行准确的分类。本章针对极化信息提取与标签噪声问题,提出了适用于极化SAR的信息提取方法与鲁棒的目标分类方法。本文针对以上问题,开展了相关的研究,主要研究内容如下:

  (1)研究了极化SAR相关理论基础。首先,介绍了电磁波极化特性的表征方式。在此基础上,介绍了目标散射特性的集中描述方法以及几种极化矩阵,并引出了极化目标分解方法的相关理论阐述,并总结了不同极化参数的物理含义。

  (2)提出了一种多类型极化信息提取方法。通过构建双通道的极化特征输入网络结构,利用空间通道注意力模块与注意力修正模块,结合跨空间学习方法,实现不同类型极化特征细化与互信息提取,并聚合不同尺寸的极化特征,为分类任务提供更全面、充分的极化信息表示方式,提升分类准确率。

  (3)提出了一种对标签噪声鲁棒的极化SAR目标分类方法。通过建立深度网络模型下噪声、干净样本损失分布差异关系,获取样本噪声概率估计,结合边界样本损失增强,实现标签噪声下鲁棒的参数优化过程,同时对边界信息充分利用,提升标签噪声下极化SAR目标分类准确率。

  使用两组真实极化SAR数据集对本文所提出方法进行有效性验证。结果表明,这些方法能够有效应对极化SAR图像目标分类中存在的主要问题,实现地物目标的高准确率分类。

  \chinesekeyword{极化合成孔径雷达,极化SAR图像,信息提取,目标分类,标签噪声}

\end{chineseabstract}