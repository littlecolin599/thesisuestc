
\begin{englishabstract}
    Synthetic Aperture Radar (SAR) is a kind of ground-imaging radar, which has the characteristics of all-weather and all-weather continuous detection of geo-targets. On the basis of SAR, Polarimetric Synthetic Aperture Radar (PolSAR) can detect the polarised scattering characteristics of ground targets by transmitting and receiving electromagnetic wave signals of different polarisation modes, which has richer polarisation information. PolSAR image interpretation technology has been widely used in military detection, disaster analysis, urban planning and other fields.

    With the continuous development of PolSAR technology, high-resolution and multi-polarised SAR image data are increasing, which provide richer feature polarisation information for complex scene analysis. The rich polarisation information and increased data volume bring positive effects to polarised SAR interpretation, but also bring new challenges. The rich polarisation information requires feature representations that can efficiently combine different types of polarisation information. In addition, the label noise introduced during the annotation process of polarised SAR datasets requires classification methods to be able to accurately classify in the presence of mislabelling. In this chapter, information extraction methods and robust target classification methods for polarised SAR are proposed to address the polarised information extraction and label noise problems.

    The main research work of this paper is as follows:

    (1) A polarisation information extractor with superior performance is necessary before conducting research on PolSAR image classification under label noise. Aiming at the problem of multi-type polarisation information characterisation, a polarisation information extraction method based on two-channel attention is proposed. The method extracts polarisation key information independently based on the difference between PolSAR scattering features and target decomposition features using the network structure of dual-channel attention. On this basis, the two types of polarisation information of dual-channel are integrated through information union to mine the complementary information between the two types of features. In addition, the cross-space feature learning method is used to aggregate different scale polarisation features. Jointly, the above methods achieve a better characterisation of the polarisation information of the feature target, thus improving the accuracy of the downstream classification task. Experiments have demonstrated that this method improves the classification task accuracy compared to using a single type of polarisation feature or directly superimposing polarisation features.

    (2) Based on the above research, a robust classification method based on mixture model estimation with boundary enhancement is proposed for the PolSAR label noise problem. The method is based on the difference between the loss of noisy samples and clean samples in the deep network model, and fits the loss distribution based on the beta mixture model to achieve the estimation of the sample noise probability. At the same time, the boundary of Pauli pseudo-color image of PolSAR is extracted and inflated based on Sobel operator, and loss enhancement is applied to the samples inside the inflated boundary. Finally, the self-learning loss function is improved to achieve robust learning optimisation by dynamically adjusting the trade-off between prediction and labelling through the sample noise probability.

    The methods proposed in this paper are validated with real PolSAR datasets, and the experimental results verify the effectiveness of these methods for target classification under polarised information extraction and label noise, and improve the accuracy of the feature target classification task.

    \englishkeyword{Polarimetric Synthetic Aperture Radar (PolSAR), PolSAR image, Polarised information extraction, target classification, label noise}
\end{englishabstract}


