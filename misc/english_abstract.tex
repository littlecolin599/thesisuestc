
\begin{englishabstract}
    Polarimetric Synthetic Aperture Radar (PolSAR) has garnered significant attention for its all-weather, all-day, and multi-polarization imaging characteristics. Classification methods for PolSAR images are crucial techniques for analyzing PolSAR data and have demonstrated important practical value in both military and civilian domains.

    As PolSAR technology evolves, the increasing availability of high-resolution, multi-polarization SAR image data has played a positive role in PolSAR image classification but has also introduced new challenges. On one hand, single-polarization features are insufficient to fully meet the diverse requirements of land cover classification, while the redundancy between multi-polarization features tends to impact classification accuracy. This necessitates target classification methods that can effectively integrate and utilize multi-polarization features. On the other hand, inevitable sample mislabeling during the PolSAR image annotation process leads to a decline in classification performance, requiring target classification methods to exhibit robustness against label noise. This thesis addresses the above issues through relevant foundational and methodological research based on deep learning methods, with the main research contents as follows:

    (1) Investigating the theoretical foundations of PolSAR. Firstly, commonly used mathematical representations of electromagnetic wave polarization characteristics are introduced. Then, two methods for describing target scattering characteristics, namely polarimetric scattering matrix and second-order statistical matrix, are derived. Finally, the principles and representative methods of polarimetric coherent and incoherent target decomposition are summarized, along with the physical meanings of different polarization parameters, laying a theoretical foundation for subsequent research.

    (2) To address the problem of decreased classification accuracy caused by redundant information from multiple types of polarized features, a polarized SAR image classification method based on dual-channel attention is proposed. By constructing a dual-channel, multi-scale feature fusion structure and utilizing the feature correlations of different polarization types, spatial positions, and feature scales in polarized SAR images, mutual information extraction and fusion of multiple types of polarized features are achieved. Subsequently, a deep classification model is constructed based on this approach, effectively improving the classification accuracy of PolSAR images.

    (3) To mitigate the decline in classification performance caused by label noise, a polarized SAR image classification method based on mixed model noise estimation is proposed. Based on the theory of finite mixture models, a loss distribution model under the mixture of noise labels and accurate labels is established to obtain the distribution estimation of noise labels in polarized SAR images. Additionally, by combining polarized pseudocolor edge information extraction methods, the classification model can utilize edge knowledge to enhance identification performance, thereby effectively improving the classification accuracy of PolSAR images under label noise.

    The proposed methods are validated using actual PolSAR data. The results indicate that the methods proposed in this thesis can effectively address the issues of integrating multi-polarization feature information and robust recognition under label noise in PolSAR image classification, thereby improving the classification accuracy of PolSAR images.

    \englishkeyword{Polarimetric Synthetic Aperture Radar (PolSAR), image classification, deep learning, polarization feature, label noise}
\end{englishabstract}


