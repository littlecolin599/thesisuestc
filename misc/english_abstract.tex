
\begin{englishabstract}
    Polarimetric Synthetic Aperture Radar (PolSAR) is an active microwave remote sensing technique known for its all-weather, all-day imaging capabilities. Due to its multi-channel, multi-polarization working characteristics, PolSAR images contain rich scattering information of targets. PolSAR image interpretation techniques have been widely applied in various fields such as military detection, disaster analysis, and urban planning.

    In recent years, the increasing availability of high-resolution, multi-polarized SAR image data has provided richer polarization information of ground targets for complex scene analysis, which has positively impacted PolSAR interpretation efforts but also introduced new challenges. Directly stacking all types of polarization information leads to information redundancy, while single-type polarization information is inadequate for all classification targets. This necessitates effective utilization of multi-type polarization information in feature representation methods. Additionally, label noise introduced during PolSAR dataset annotation requires classification methods to accurately classify samples with partial mislabeling. This thesis addresses the aforementioned issues through relevant fundamental research and innovations, with the following main research contents:

    (1) Investigation of PolSAR-related theoretical foundations. Firstly, characterization methods of electromagnetic wave polarization properties are introduced. Subsequently, several methods for describing target scattering characteristics and various polarization matrices are discussed. Finally, the theoretical explanation of polarization target decomposition methods is presented, along with a summary of the physical meanings of different polarization parameters.

    (2) Addressing the problem of redundancy in multi-type polarization feature information, a PolSAR target classification method based on dual-channel attention is proposed. By constructing a dual-channel PolSAR feature input network structure and employing spatial and channel attention modules, as well as multi-scale learning methods, different types of polarization features are refined and mutual information is extracted. Moreover, aggregation of polarization features at different scales is performed, effectively avoiding the limitations of single features and reducing the redundancy of multi-feature information, thereby improving PolSAR target classification accuracy.

    (3) In response to label noise issues, a robust PolSAR target classification method under label noise is proposed. By establishing the relationship between the loss distribution differences of noisy and clean samples under a deep neural network model, the noise probability estimation of samples is obtained. Additionally, combined with boundary sample loss enhancement, robust parameter optimization under label noise is achieved. Furthermore, full utilization of boundary information is implemented, resulting in improved PolSAR target classification accuracy under label noise.

    The effectiveness and superiority of the proposed methods are validated using real PolSAR data. Results demonstrate that these methods effectively address the problems of redundancy in multi-polarization feature information and label noise in PolSAR image classification, achieving high-accuracy classification of ground targets.

    \englishkeyword{Polarimetric Synthetic Aperture Radar (PolSAR), PolSAR image, polarization feature, target classification, label noise}
\end{englishabstract}


