\chapter{全文总结与展望}
\section{全文总结}
极化合成孔径雷达通过收发不同极化方式的电磁波对地物目标进行探测,提供了丰富的目标散射信息,广泛应用于军事、民事等领域中。极化SAR图像解译作为极化SAR应用的关键步骤,是当前研究的重点方向。本文以极化SAR图像目标分类为主要研究问题,展开了对极化特征表示与标签噪声目标分类的研究。极化SAR图像蕴含丰富的极化信息,不同类型的极化信息从不同角度描述地物目标散射特性,本文提出了双通道注意力的极化信息提取方法,对散射特征和目标分解特征进行联合,提升极化信息的可表达性。针对极化SAR图像分类中标签噪声问题,本文提出了混合模型与边界增强的分类方法,通过对样本损失分布建模,实现噪声样本的概率估计,通过鲁棒性分类方法联合极化信息提取器,在多组实验数据中证实有效提升了分类精度。具体的研究内容如下:

(1)将极化SAR图像中常用的散射特征与目标分解特征相结合,设计了双通道注意力的极化信息提取网络结构。该方法利用双通道的网络结构,基于注意力方法,实现对散射特征和目标分解特征中关键极化信息的捕获。利用联合注意力调整方法,实现对两类注意力参数的调整,结合不同类型的极化信息。基于多尺度学习方法,增强对不同空间尺寸的极化信息的感知。最后在两个不同的极化SAR数据集上进行验证对比。

(2)在上一个工作基础上,针对极化SAR图像分类中的标签噪声问题,提出了混合模型与边界增强的鲁棒性分类方法。该方法首先通过混合模型对样本损失分布进行拟合,不断更新迭代参数,利用模型参数估计样本噪声概率。同时,为了提升模型对边界信息的利用,增强对边界重要样本的感知能力,通过边界提取与膨胀操作,对膨胀边界内的训练样本进行损失增强。最后,改进自学习损失函数优化方法,通过噪声概率实现对模型预测与样本标签的依赖调整,实现鲁棒训练过程,最终提升分类精度。在两个不同的数据集上,分别验证了对称噪声与非对称噪声下的准确率精度。

\section{后续工作展望}
极化SAR图像信息提取与目标分类研究是目前极化SAR图像解译的关键步骤,创新方法层出不穷,一直是研究的热点与难点。

极化SAR图像信息提取与目标分类研究是一个设计多个跨学科交叉领域的遥感图像处理技术,包括雷达成像、微波遥感、模式识别和机器学习等。相关研究的创新方法层出不穷,一直都是研究的热点与难点。本文针对极化信息提取与标签噪声目标分类问题展开研究,提出一些新的解决思路与方案,但是依然存在许多问题亟待解决,有待在将来的工作中进一步完善,主要包括:

(1)极化信息提取方面:

本文第三章针对极化SAR图像的散射特征和目标分解特征的综合利用展开了研究,旨在对不同类型的极化特征进行有效联合,利用双通道网络结构与注意力调整方法实现独立特征与联合特征的捕捉。该方法尽管考虑了对散射特征与目标分解特征的差异与联系,而极化SAR图像中还存在空间特征、颜色特征、纹理特征等。因此,设计更加全面的极化信息提取器,构建全面、高效的极化信息融合方法,对于极化SAR图像解译具有支撑性作用。

(2)目标分类方面:

本文提出的极化SAR目标分类方法是以单个像素为单位,而通常极化SAR数据集数量巨大,逐个对极化SAR像素进行分类可能会导致计算量大、计算时间长的问题。因此,在后面的工作中,可以在减小时间复杂度的方面考虑,设计更加快速、准确的极化SAR图像分类方法。

(3)噪声标签分类方面:

本文第四章通过混合模型与边界增强,实现了对标签噪声具有鲁棒性的分类方法。但是该方法的分类损失模型仅是通过调整预测标签与真值标签的权衡,并没有完全消除标签噪声被学习的过程。因此,在明确标签噪声学习理论的基础上,构建更加高效的目标损失函数,对于提升标签噪声样本学习模型的性能具有重要意义。


